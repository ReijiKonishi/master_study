%!TEX root = ../thesis.tex

\subsection{研究目的}

本研究の目的は、統計的因果探索の手法を応用し、
マーケティング・リサーチにおいて消費者行動の因果構造に関する仮説を
構築する手法の開発を行うことである。

マーケティング・リサーチで扱う定量データは、先述のようにアンケート調査データや、
購買履歴データ、アクセスログなどが挙げられる。
アンケート調査データには、聴取内容に応じて様々な尺度のデータが含まれており、
離散変数と連続変数の両方が混在している。
例えば、「商品Aの認知」については「知っている/知らない」
といった2値(0,1)で扱われる。
一方で、「商品Aの購入意向(買いたい気持ち)」については
「買いたい/やや買いたい/どちらとも言えない/あまり買いたくない/買いたくない」
といった5段階尺度で聴取される。
このような5段階などの尺度で聴取されたデータは、連続値として扱われることが多い。
また、ID付きPOSデータやアクセスログはトランザクションデータであるため、
マーケティング・リサーチで分析を行う際は、
商品やWebページ、ユーザー単位で集計を行う。
例えば、「ユーザーXは1年間に商品Aを10個、商品Bを5個買った」
などと集計されたデータで分析を行う。
このような購買個数などのデータは0以上の整数を取るカウント(計数)データであるため、
連続値として扱うことはできず、ポアソン分布などの離散確率分布を用いて分析する必要がある。

しかし、既存の統計的因果探索の手法は、
連続変数か離散変数のどちらか一方のみに限定されたモデルが中心である。
また、連続変数と離散変数が混在するモデルも一部提案されているが、
扱える確率分布が限定的であったり、離散変数が2値(0,1)のみであったりする。
そのため、従来手法では
マーケティング・リサーチで扱う様々なデータにおける
因果構造の探索が難しいという課題が残っている。
そこで本論文では、
マーケティング・リサーチの分野で活用することを念頭に、
離散変数と連続変数の両方が混在する構造的因果モデルを提案し、
その因果構造の識別可能性について議論する。
本研究では、ポアソン分布などのカウントデータを扱うことができる
2次分散関数(QVF)DAGモデル\cite{Park2017-hw}と
連続変数を扱うことができるAdditive Noise Model\cite{Park2020-ey}を
組み合わせることで、
マーケティング・リサーチで扱う定量データにおける因果順序と因果構造の推定を行う。


%さらに、市場調査会社などでは、
%アンケート調査データや購買履歴データ、アクセスログなどを
%シングルソース
%\footnote{基本属性や購買行動などこれまで別々の情報源から得られることが多かった情報が
%同じ消費者に対して得られているようなデータのこと\cite{2018-ci}。}
%で取得している場合もある。
%意識データと行動データをシングルソースで扱うことで、
%購買行動などの事実だけでなくその行動を取ったときの意識・理由などを同時に
%分析することができ、消費者行動の因果構造の理解に活用されている。
