%!TEX root = ../thesis.tex

\subsection{研究目的}

本研究の目的は、統計的因果探索の手法を応用し、
マーケティング・リサーチにおいて消費者行動の因果構造に関する仮説を
構築する手法の開発を行うことである。
マーケティング・リサーチで扱う定量データは、先述のようにアンケート調査データや、
POSデータ、ID付きPOSデータなどが中心である。
特にアンケート調査データには、聴取内容に応じて様々な尺度のデータが含まれており、
離散変数と連続変数の両方が混在している。

\textcolor{red}{以下のような例を表にまとめる}

\begin{itemize}
  \item 知ってる/知らない などの01データ
  \item 最もよく買うブランドはどれか? などの名義尺度(ダミー変数に変換して分析)
  \item あてはまる〜あてはまらない のようなリッカート尺度(連続変数とみなすことが多い)
  \item 購買頻度(毎日買う/週1以上/月1以上)のような順序尺度(もともとはカウントデータ)
\end{itemize}

そこで、本論文では離散変数と連続変数の両方が含まれる構造的因果モデルを提案し、
その因果構造の識別可能性を議論する。
