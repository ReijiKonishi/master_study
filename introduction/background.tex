%!TEX root = ../thesis.tex

\subsection{研究背景}

企業は自社の商品やサービスを顧客に提供するために、様々なマーケティング活動を行っている。
近年では消費者の嗜好が多様化したり、新型コロナウイルス感染症が流行したりするなど、
企業活動を取り囲む環境が日々大きく変化しており、
企業はその環境変化に適応する必要がある。
それ故、商品・サービスの開発や消費者とのコミュニケーションなどのマーケティング活動を
適切に実行するためには、消費者の行動について深く理解することがより一層重要となっている。

アメリカ・マーケティング協会(AMA)によると、
マーケティングの定義は「顧客、クライアント、パートナー、社会のための
価値の創造、伝達、提供、交換という全体の活動」であり、
マーケティング・リサーチは「消費者、顧客、公衆とマーケターが情報を介してつながる機能」である
と定義している
\footnote{https://www.ama.org/the-definition-of-marketing-what-is-marketing/}。
また、その具体的な業務として、
「必要な情報を特定し、情報収集のための方法を設計し、データ収集プロセスを管理・実施し、
結果を分析し、分析結果と結果から得られる示唆伝えること」としている。
つまり、上記の定義を合わせて考えると、マーケティング・リサーチは
「マーケティング課題の発見や施策の実行に必要なデータを適切に収集し、
得られたデータを分析することで、企業のマーケティング活動を支援すること」と理解することができる。

企業のマーケティング活動は、主に以下の4つのフェーズに分類することが可能で、
商品・サービスのカテゴリにも依存するが、約1〜数年程度の期間で繰り返されることが一般的である。
この繰り返しのことをマーケティング・サイクルと呼ぶ。
\begin{itemize}
  \item 市場機会の発見
  \item コンセプト開発
  \item コミュニケーション内容・販売施策の策定
  \item 施策後の効果検証
\end{itemize}
マーケティング・リサーチは、各フェーズにおけるマーケティング担当者の関心事に対して、
適切な示唆を与えることが求められている。
%コトラーは「誰に対してどのような価値を提供するのか」というマーケティングの目的を達成するために、
%消費者を細分化(セグメンテーション)し、
%その中でどのセグメントを優先的に自社の顧客とするのか(ターゲティング)を明確にし、
%他商品との関係性を設定すること(ポジショニング)が必要不可欠だと指摘している\cite{Kotler1999-im}。
「市場機会の発見」では、
「市場にある商品で満たされていないニーズは何か?」や
「この商品を購入している人はどのような特徴があるのか?」といったことが、
マーケティング担当者の関心事となり、
未充足ニーズの探索や、消費者セグメントの整理などがリサーチの役割となる。
次に「コンセプト開発」では、
「どのような価値を提供すると売れるのか?」や
「ターゲットとなる消費者の規模はどのくらいか?」といったことが関心事となり、
コンセプトの受容性確認や、消費者セグメントの規模感把握などがリサーチに求められる。
また「コミュニケーション内容・販売施策の策定」では、
「商品の特長をどのように伝えると購買に結びつくのか?」や
「商品パッケージや価格はどのようにすればよいか?」といったことが主な関心事となり、
広告内容の精査や、商品の改善点抽出などが行われる。
最後に「施策後の効果検証」においては、
市場浸透度の確認や、広告量などの投資に対する売上の費用対効果などが行われる。
各フェーズでのマーケティング担当者の関心事を俯瞰すると、
多くは「マーケティング活動と消費者行動の因果関係」にあると言える。

「マーケティング」の他に「ブランディング」も類似の意味を持つ言葉として頻繁に用いられるが、
それぞれの言葉の意味には異なる部分が存在する。
音部(2019)\cite{2019-eb}によると、マーケティングは属性の順位を変換して市場を創造することを目指し、
結果的にニーズを作り出すことにつながっており、
ブランディングはブランドの意味の確立を目指し、結果的にベネフィットを作り出すことにつながっている。
つまり、マーケティングは消費者行動の因果構造そのものを変化させることによって、
自社の商品が有利に購買されるような状況を作り出す活動であることに対し、
ブランディングは現在の消費者行動の因果構造はそのままに、
自社の商品に対する消費者の認識を変化させ、自社の商品を購買することの必然性を高める活動であると
捉えることができる。

このように企業は、自社の商品やサービスを顧客に効率的に提供するために、
マーケティング・リサーチを通じて消費者に関する情報を収集・分析し、
消費者行動の因果関係について仮説を立てたり解釈を行ったりしている。
そのため、マーケティング・リサーチにおいて、消費者行動の因果関係に関する情報を得る手段は、
非常に重要な役割を担っている。
