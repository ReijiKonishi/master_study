%!TEX root = ../thesis.tex

\subsection{本論文の構成}

まず2章では、
本研究の基礎となる統計的因果探索の従来研究として、
連続変数を扱うANMと
離散変数を扱うQVF DAGモデルについて述べる。
3章では、Park and Park(2019)\cite{Park2019-qy}のアイデアを用いて
QVF DAGモデルの識別可能性を証明する。
4章では、ANMとQVF DAGモデルを用いることによって、
連続変数と離散変数が混在する構造的因果モデルを提案し、その識別可能性を証明する。
5章では、4章で提案したモデルを推定する手法について述べる。
6章では、数値実験によって、予め設定した因果構造に基づいてデータを発生させ、
その因果構造の推定を行う。
数値実験の結果、従来手法と比較して提案手法が有効であることを示す。
最後に7章では、本論文のまとめと今後の課題について述べる。
