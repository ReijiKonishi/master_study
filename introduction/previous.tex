%!TEX root = ../thesis.tex

\subsection{既存手法・研究}

因果関係に関する考察は、科学の基本的な問いとして因果推論という文脈で研究されてきた。
近年では計算機科学や統計科学の発展により、経済学やマーケティングの分野でも注目されている\cite{Varian2016-se}。
因果推論は大まかには、
(i)因果構造の同定、と
(ii)因果構造を既知としたその因果関係の大きさの推定、
という2つの問題に分類できる。
(i)因果構造の同定 は、非常に困難な問題であるとされているが、
いくつかの仮定の上で識別可能なモデルが提案されている。
(ii)因果構造を既知としたその因果関係の大きさの推定 は、
ランダム化比較試験を中心とした実験研究と、
特に実験を行わない観察研究の双方において統計科学の文脈で研究されている。

本節では、マーケティング・リサーチにおいて消費者行動の因果関係に関する情報を得る方法として
従来より広く用いられてきた一般的な既存手法について述べた後、
統計的因果探索と呼ばれる因果構造の同定に関する近年の研究について俯瞰する。

\subsubsection{マーケティング・リサーチにおける既存手法}

マーケティング・リサーチにおいて消費者行動の因果関係に関する情報を得る方法は、
大きく定性調査と定量調査に分類される。

定性調査は一般的に、消費者の内部にある意見や態度を理解することで、
マーケティング課題の詳細な定義、仮説の設定、調査項目の優先順位の決定、
消費者独自の考え方や表現の理解、企業のマーケティング担当者の不足している知識の吸収、
定量調査における最重要な項目に関する示唆を得るなどの目的で実施される\cite{2018-ci}。
主に深層面接法(デプス・インタビュー)や集団面接法(グループ・インタビュー)といった、
インタビュアーが回答者との対話を通じて質問をし回答を得る方法が一般的である。
そのため、調査対象者が日常生活であまり気に留めていないことや、
誰かに問いかけられて初めて気づくことなどを収集することができ\cite{2018-ci}、
マーケティング担当者の周辺知識では想定しきれなかった因果関係を発見できる可能性がある。
一方で、定性調査は定量調査と比べて一般的に時間や費用が多くかかることや、
定量的な評価ができず、得られた意見や行動の一般性・代表性に関する議論ができないことなどの
デメリットがある。


定量調査では、主にアンケート調査に代表される意識データや、
POSデータ、ID付きPOSデータ、位置情報などの行動データが用いられる。
これらのデータの分析手法は非常に多岐に渡るが、
消費者行動の因果関係を評価する手法としては、
一般化線形モデル(generalized linear model, GLM)や
構造方程式モデル(structural equation model, SEM)を用いることが多い\cite{2015-pb}\cite{2009-qw}。
GLMでは、目的変数と説明変数の関係を定式化し、
各係数を目的変数に対する説明変数の因果関係の大きさとして解釈を行うことが慣例となっている。
しかし、宮川(2004)\cite{2004-qj}で述べられている通り、
GLMは説明変数を与えたときの目的変数の条件付き確率分布に関するモデルであり、
多変量の相関関係を利用しているに過ぎず、
その係数に対して因果的な解釈を行うことは誤りとなる可能性がある。
一方で、SEMはデータの生成過程を記述した統計的因果モデルであり、
その係数は単なる相関関係の尺度ではなく、因果的な解釈を行うことができる\cite{2004-qj}。
ただし、SEMは分析者の事前知識を積極的に利用することで因果構造の仮説を有向グラフで表現した上で、
その構造に対してモデリングを行う手法である。
つまり、SEMは上述の(ii)因果構造を既知としたその因果関係の大きさの推定を行う手法であると言える。
そのため、因果構造の仮説構築は分析者側の事前知識の質や量に依存し、
妥当だと思われるモデルを得るまでに長い時間を要したりしているという現状がある。
また、近年の社会の発展により消費者の行動は複雑化しているため、
因果構造に関する仮説を構築することが難しい場合も少なくない。


\subsubsection{統計的因果探索}

統計的因果探索とは、因果構造が未知である際に、
どのような条件で観測データから因果構造が復元することが可能であるかを明らかにする学問分野である。

\textcolor{red}{以下の内容あたりをまとめる。}
\begin{itemize}
  \item 忠実性を仮定すると、条件付き独立関係からマルコフ同値類まで探索することができる
  \item 関数形や誤差分布などに仮定を置くと、因果構造が一意に復元可能なモデルが提案されている(LiNGAMなどのANM)
  \item ほとんどの既存手法は、データがすべて離散かすべて連続を仮定している
  \item カウントデータを仮定したモデルとしてPark and Raskutti(2015)\cite{Park2015-tj}
  \item それを拡張する方法などとして、Park and Raskutti(2017)\cite{Park2017-hw}や、
        Park and Park(2019)\cite{Park2019-qy}
  \item 連続・離散混在のモデルとして、Wenjuan \textit{et al.}(2018)\cite{Wenjuan2018-nm} や、
        Li and Shimizu(2018)\cite{Li2018-aw}や、
        Yamayoshi \textit{et al.}(2020)\cite{Yamayoshi2020-gu}など
\end{itemize}
