%!TEX root = ../thesis.tex

\section{結論}
\label{part:conclusion}

マーケティング・リサーチでは消費者行動の因果関係に関する情報を得るために、
定性調査や定量調査などが行われ、様々な分析手法が用いられてきた。
通常、因果関係を評価するためにはランダム化比較試験と中心とした実験研究が必要であるが、
実験を行うことが難しい場合や非常に多くのコストがかかる場合が多い。
そのため、観察研究によって因果関係を評価する必要があるが、
観察データでは相関関係しか得られないため、
因果効果の大きさの推定や因果構造の復元は困難である。
しかし、統計的因果探索の分野の研究成果により、
観察データから因果構造を復元できる条件や方法が明らかとなっている。

%!TEX root = ../thesis.tex

\subsection{本研究の成果}

本研究では、従来から提案されている統計的因果探索のモデル・手法を基礎に、
マーケティング・リサーチで扱う離散変数と連続変数の両方が混在するモデルを提案し、
その識別可能条件と推定アルゴリズムを示した。
具体的には、連続変数のデータ生成過程を親変数と誤差変数の線形和で表現し、
離散変数のデータ生成過程を一般化線形モデル(GLM)で表現し、
連続変数と離散変数が混在するモデルを提案した。
ただし、離散変数の生成過程は、2次分散関数性を満たす確率分布
(分散が期待値の2次式で表せる分布)に従うものとした。
2次分散関数性を満たす確率分布には、二項分布やポアソン分布、負の二項分布などが含まれており、
商品の購買個数や購入回数などマーケティング・リサーチで扱う
カウントデータを扱うことが可能である。
また、連続変数が割り当てられた任意の頂点と
その子孫のうち連続変数が割り当てられた頂点のデータ生成過程における誤差変数の分散について、
定理\ref{theo:prop_identifiability}の仮定(A)の大小関係が成立する場合、
提案モデルは識別可能であることを証明した。
更に、提案モデルが識別可能条件を満たす時に、
因果順序やDAGの構造、モデルのパラメータを推定するアルゴリズムを提案した。

数値実験では、識別可能な提案モデルに従ってデータを生成し、
提案アルゴリズムと既存アルゴリズムの推定精度の比較を行った。
その結果、提案モデルは経験的にも識別可能であり、
提案アルゴリズムは既存アルゴリズムよりも推定精度が高いことを示した。

提案アルゴリズムには、離散変数と連続変数の因果順序を求める際の閾値に関する
調整パラメータが存在する。
そのため、閾値の設定によっては因果順序やDAGの推定精度に影響を与えることが考えられる。
そこで数値実験を行うことで、範囲$[1.01, 1.02]$で閾値を設定すると
因果順序の推定精度が高くなることを示した。


%!TEX root = ../thesis.tex

\subsection{マーケティング・リサーチにおける統計的因果探索の応用法}

本節では、マーケティング・リサーチにおいて
本論文で提案したモデルを含む統計的因果探索を活用する方法について述べる。


%!TEX root = ../thesis.tex

\subsection{今後の課題}

最後に今後の課題について述べる。
まず、本論文でで提案したモデル・手法を
実際のマーケティング・リサーチで得られたデータに適用することで、
有用な因果仮説が得られるかどうかを確認することである。
ただし、「有用な因果仮説」は企業やブランドのマーケティング課題によって変化するため、
単純に背景理論と合致するかどうかなどで評価することはできないと考えられる。
そのため、実際にマーケティング・リサーチの実務において、
提案モデルを用いて因果仮説構築を行った事例を重ね、
実務家からの意見を聞くことが必要である。

理論的な視点では、
識別可能条件が成立していることを検証する手法の開発や、
識別可能条件の緩和が挙げられる。
本論文で示した提案モデルの識別可能条件は、
連続変数のデータ生成過程における誤差変数の分散の大小に関する強い条件である。
通常、データの分散は観測したデータのスケールに依存する。
そのため、観測したデータの単位を変えるだけで、
識別可能性が変化する可能性がある。
さらに、観測データにおいてこの条件が成立しているかどうかを検証する方法は、
筆者の知る限りでは存在しない。

次に、ゼロ過剰ポアソン分布などのゼロが多く含まれるカウントデータを扱うモデルの開発が挙げられる。
例えば、購買データにおける「ある商品Aの購買回数」といったデータには、
0回というデータが非常に多く含まれ、
0回の人と1回以上の人ではデータ生成過程が異なると考えられる。
0回の人は商品Aが属するカテゴリを購買しない人である場合や、
近隣の店頭で販売されていないなどで入手できない人である場合などである。
マーケティング・リサーチにおける行動データにはこのようなデータが多いため、
行動データに関する因果構造を推定するためには、
ゼロ過剰ポアソン分布などを扱えるようにモデルを拡張する必要がある。

最後に、未観測共通原因がある場合でも適用可能な
連続・離散混合モデルの研究が挙げられる。
通常、未観測共通原因が存在する観測変数間の因果構造を推定することはできない。
しかし、マーケティング・リサーチでは、
人間の行動をアンケート調査データや購買データなどによって観測しているため、
未観測な事項が多い。
例えば、消費者が商品を選ぶまでに触れた広告や情報、商品を選んだ時の感情などは
通常観測できない。
そのため、今後マーケティング・リサーチにおいて統計的因果探索の手法をより活用するためには、
未観測共通原因が存在しないという仮定を緩和する必要がある。

