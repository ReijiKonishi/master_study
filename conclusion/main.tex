%!TEX root = ../thesis.tex

\section{結論}
\label{part:conclusion}

マーケティング・リサーチでは消費者行動の因果関係に関する情報を得るために、
定性調査や定量調査などが行われ、様々な分析手法が用いられてきた。
通常、因果関係を評価するためにはランダム化比較試験を中心とした実験研究が必要であるが、
実験を行うことが難しい場合や非常に多くのコストがかかる場合が多い。
そのため、観察研究によって因果関係を評価する必要があるが、
観察データでは相関関係しか得られないため、
因果効果の大きさの推定や因果構造の復元は困難である。
しかし、統計的因果探索の分野の研究成果により、
観察データから因果構造を復元できる条件や方法が明らかとなっている。

%!TEX root = ../thesis.tex

\subsection{本研究の成果}

本研究では、従来から提案されている統計的因果探索のモデル・手法を基礎に、
マーケティング・リサーチで扱う離散変数と連続変数の両方が混在するモデルを提案し、
その識別可能条件と推定アルゴリズムを示した。
具体的には、連続変数のデータ生成過程を親変数と誤差変数の線形和で表現し、
離散変数のデータ生成過程を一般化線形モデル(GLM)で表現た。
ただし、離散変数の生成過程は、2次分散関数性を満たす確率分布
(分散が期待値の2次式で表せる分布)に従うものとした。
2次分散関数性を満たす確率分布には、二項分布やポアソン分布、負の二項分布などが含まれており、
商品の購買個数や購入回数などマーケティング・リサーチで扱う
カウントデータを扱うことが可能である。
また、連続変数が割り当てられた任意の頂点と
その子孫のうち連続変数が割り当てられた頂点のデータ生成過程における誤差変数の分散について、
定理\ref{theo:prop_identifiability}の仮定(A)の大小関係が成立する場合、
提案モデルは識別可能であることを証明した。
更に、提案モデルが識別可能条件を満たす時に、
因果順序やDAGの構造、モデルのパラメータを推定するアルゴリズムを提案した。

数値実験では、識別可能な提案モデルに従ってデータを生成し、
提案アルゴリズムと既存アルゴリズムの推定精度の比較を行った。
その結果、提案モデルは経験的にも識別可能であり、
提案アルゴリズムは既存アルゴリズムよりも推定精度が高いことを示した。

提案アルゴリズムには、離散変数と連続変数の因果順序を求める際の閾値に関する
調整パラメータが存在する。
そのため、閾値の設定によっては因果順序やDAGの推定精度に影響を与えることが考えられる。
そこで数値実験を行うことで、範囲$[1.01, 1.02]$で閾値を設定すると
因果順序の推定精度が高くなることを示した。


%!TEX root = ../thesis.tex

\subsection{マーケティング・リサーチにおける統計的因果探索の応用法}

本節では、マーケティング・リサーチにおいて
本論文で提案したモデルを含む統計的因果探索を活用する方法について述べる。


%!TEX root = ../thesis.tex

\subsection{今後の課題}

今後の課題だよ〜

