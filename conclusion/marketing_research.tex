%!TEX root = ../thesis.tex

\subsection{マーケティング・リサーチにおける統計的因果探索の応用法}

本節では、マーケティング・リサーチにおいて
本論文で提案したモデルを含む統計的因果探索を活用する方法について述べる。
マーケティング・リサーチでは、企業のマーケティング課題に合わせて
調査課題を設定し、リサーチの設計、データの収集・分析が行われる。
「データ分析」の主な目的は、
「調査課題の設定」の際に立てた仮説を検証することであるが、
消費者行動に関する因果構造は複雑であるため、
仮説が明確になっていないことや仮説を絞りきれていないことが多い。
そのため、仮説検証型の分析だけでなく仮説探索型の分析も行う必要がある。
仮説探索型の分析では、データにおける相関関係などを確認しながら、
分析者の経験や事前知識によって「検証すべき因果仮説」を構築していく。
しかし、分析者の事前知識が不十分である場合や
擬似相関が多く含まれる場合など、分析に非常に長い時間がかかる。
そこで、本論文で提案したモデルを含む統計的因果探索を用いると、
観測データにおける因果構造がDAGとして推定されるため、
推定されたDAGを元に効率的に因果仮説を構築することが可能になると考えられる。
また、DAGは多変量のデータの関係性を直感的に表現できるため、
分析において仮定している因果構造の認識を
リサーチに関わるメンバー間で共有することが容易になと考えられる。

しかし、マーケティング・リサーチにおいて統計的因果探索を利用することには注意すべき点もある。
1つ目は、本論文の提案モデルを含む多くの統計的因果探索の手法では、
未観測共通原因がないことを仮定していることである。
つまり、捉えようとしている因果構造に関係する変数を網羅的に観測している必要がある。
そのため、「リサーチの設計」の際にリサーチに関わるメンバー同士で、
どのようなデータを聴取・収集する必要があるかを十分に議論する必要があると言える。
2つ目は、統計的因果探索が必ず真の因果構造を推定できるとは限らないことである。
例えば、実際のデータを統計的因果探索を用いて分析すると、
マーケティングの背景理論と合致しないDAGが推定される可能性がある。
先述の通り統計的因果探索は、データ生成過程に対する条件が成立しているという仮定の下で、
元の因果構造を推測する手法である。
そのため、データ生成過程に対する条件が成立していなければ、
因果構造を復元することは保証されていない。
そういった問題を解決するための手段の1つとして、
因果構造を推定する際に背景知識を取り入れて推定を行うことが挙げられる。
「変数$X$は変数$Y$の直接的な原因にはならない」などの背景知識を取り入れて
因果順序を推定することなどが提案されている\cite{Shimizu2011-pd}。
また、背景理論と合致しない場合、
どの部分が合致していないのか、なぜ合致していないのかなどを
推定されたDAGを用いて議論することも
消費者行動に関する因果仮説を構築するために重要なことであると言える。
