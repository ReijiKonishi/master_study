%!TEX root = ../thesis.tex

\subsection{本研究の成果}

本論文では、従来から提案されている統計的因果探索のモデル・手法を基礎に、
マーケティング・リサーチで扱う離散変数と連続変数の両方が混在するモデルを提案し、
その識別可能条件と推定アルゴリズムを示した。
具体的には、連続変数のデータ生成過程を親変数と誤差変数の線形和で表現し、
離散変数のデータ生成過程を一般化線形モデル(GLM)で表現し、
連続変数と離散変数が混在するモデルを提案した。
ただし、離散変数の生成過程は、2次分散関数性を満たす確率分布
(分散が期待値の2次式で表せる分布)に従うものとした。
2次分散関数性を満たす確率分布には、二項分布やポアソン分布、負の二項分布などが含まれており、
商品の購買個数や購入回数などマーケティング・リサーチで扱う
カウントデータを扱うことが可能である。
また、連続変数が割り当てられた任意の頂点と
その子孫のうち連続変数が割り当てられた頂点のデータ生成過程における誤差変数の分散について、
定理\ref{theo:prop_identifiability}の仮定(A)の大小関係が成立する場合、
提案モデルは識別可能であることを証明した。
数値実験
