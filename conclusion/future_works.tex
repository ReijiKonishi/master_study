%!TEX root = ../thesis.tex

\subsection{今後の課題}

最後に今後の課題について述べる。
まず、本論文で提案したモデル・手法を
実際のマーケティング・リサーチで得られたデータに適用することで、
有用な因果仮説が得られるかどうかを確認することである。
ただし、「有用な因果仮説」は企業やブランドのマーケティング課題によって変化するため、
単純に背景理論と合致するかどうかなどで評価することはできないと考えられる。
そのため、実際にマーケティング・リサーチの実務において、
提案モデルを用いて因果仮説構築を行った事例を重ね、
実務家からの意見を聞くことが必要である。

理論的な視点では、
識別可能条件が成立していることを検証する手法の開発や、
識別可能条件の緩和が挙げられる。
本論文で示した提案モデルの識別可能条件は、
連続変数のデータ生成過程における誤差変数の分散の大小に関する強い条件である。
通常、データの分散は観測したデータのスケールに依存する。
そのため、観測したデータの単位を変えるだけで、
識別可能性が変化する可能性がある。
さらに、観測データにおいてこの条件が成立しているかどうかを検証する方法は、
筆者の知る限りでは存在しない。

次に、ゼロ過剰ポアソン分布などのゼロが多く含まれるカウントデータを扱うモデルの開発が挙げられる。
例えば、購買データにおける「ある商品Aの購買回数」といったデータには、
0回というデータが非常に多く含まれ、
0回の人と1回以上の人ではデータ生成過程が異なると考えられる。
0回の人は、商品Aが属するカテゴリを購買しない人である場合や、
近隣の店頭で販売されていないなどで入手できない人である場合などである。
マーケティング・リサーチにおける行動データにはこのようなデータが多いため、
ゼロ過剰ポアソン分布などを扱えるようにモデルを拡張することで、
行動データに関する因果構造の推定などにおける活用の機会が広がると考えられる。

最後に、未観測共通原因がある場合でも適用可能な
連続・離散混合モデルの研究が挙げられる。
通常、未観測共通原因が存在する観測変数間の因果構造を推定することは困難である。
しかし、マーケティング・リサーチでは、
人間の行動をアンケート調査データや購買データなどによって観測しているため、
未観測な事項が多い。
例えば、消費者が商品を選ぶまでに触れた広告や情報、商品を選んだ時の感情などは
通常観測できない。
そのため、今後マーケティング・リサーチにおいて統計的因果探索の手法をより活用するためには、
未観測共通原因が存在しないという仮定を緩和する必要がある。
