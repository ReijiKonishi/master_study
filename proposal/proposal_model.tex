%!TEX root = ../thesis.tex

\subsection{提案モデル}

提案モデルにおける変数は、離散変数と連続変数に分けられ、
離散変数は0以上の整数を取る確率変数であると仮定する。
そこで提案モデルを以下のように定義する。

\begin{enumerate}
  \item
  $p$個の観測変数$X = \{ X_1, \dots, X_p \}$は
  DAGによって表現されるデータ生成仮定から生成されており、
  各変数の親変数がその変数の直接的な原因である。

  \item
  連続変数に割り当てられた変数$X_j$は、
  その親変数$Pa(j)$と誤差変数$e_j$の線形和である。

  \begin{equation}
    X_j = e_j + \theta_{j} + \sum_{k \in Pa(j)} \theta_{jk}X_j
    \quad \text{with} \quad e_j \sim \mathit{Laplace}(0, b_j)
    \label{eq:lingam_prop}
  \end{equation}

  それぞれの係数$\theta_{jk}$は、変数$X_k$から変数$X_j$への直接的な因果効果の大きさを表す。
  また、誤差変数$e_j$はラプラス分布に従う確率変数であり、お互いに独立である。

  \item
  離散変数に割り当てられた変数$X_j$は、
  その親変数$Pa(j)$による条件付き確率が、2次分散関数性を満たす。
  つまり、以下を満たすような$\beta_{j0},\beta_{j1} \in \mathbb{R}$が存在する。

  \begin{equation}
    \mathit{Var}(X_j|X_{Pa(j)}) = \beta_{j0} E(X_j | X_{Pa(j)}) + \beta_{j1} E(X_j | X_{Pa(j)})^2
    \label{QVF_prop}
  \end{equation}

  また、各変数の条件付き期待値は、
  その変数の親変数$Pa(j)$と
  任意の単調で微分可能なリンク関数$g_j \colon \mathcal X_{Pa(j)} \rightarrow \mathbb R^+$
  によって以下のように記述される。
  \begin{equation}
    E(X_j | X_{Pa(j)})
    = g_j(X_{Pa(j)})
    = g_j \left(\theta_j + \sum_{k \in Pa(j)} \theta_{jk}X_k \right)
  \end{equation}
\end{enumerate}
