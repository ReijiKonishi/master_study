%!TEX root = ../thesis.tex

\subsection{数学的準備}

グラフは頂点(node)の集合$V=\{1,2,\dots,p\}$と、
頂点同士をつなぐ辺(edge)の集合$E\subset V \times V$ によって、
$G=(V,E)$と表現される。
グラフの辺は有向辺(矢線)と無向辺(双方向矢線)に分けることができ、
2つの頂点$j,k\in V$において、
$(j,k)\in E$かつ$(k,j)\notin E$のとき、$j$から$k$への矢線があるという。
これを$j \rightarrow k$と表現することもある。
一方で、$(j,k)\in E$かつ$(k,j)\in E$のとき、$j$と$k$の間に双方向矢線があるという。
すべての辺が有向辺であるグラフを有向グラフ(directed graph)という。
本論文では、特に断りのない限り、頂点$j$から$k$への矢線がある場合、
$j$が$k$の原因であるといった因果関係があることを表す。
つまり、本論文で扱うグラフにおける矢線の有無は因果関係の有無を表しており、
矢線の始点が原因で、矢線の終点が結果である。
このような定性的な因果関係を表すグラフを因果グラフ(causal graph)という。
また、グラフ$G$からすべての矢印を取り除くことによって得られるグラフを
$G$のスケルトンという。

頂点の系列$\alpha_1, \alpha_2, \dots, \alpha_{n+1}$について、
すべての$i=1,2,\dots, n$で、
$\alpha_i \rightarrow \alpha_{i+1}$、または$\alpha_{i+1} \rightarrow \alpha_i$となる矢線がある時、
長さ$n$の道(path)という。
特に、すべての$i=1,2,\dots, n$で、$\alpha_i \rightarrow \alpha_{i+1}$となる矢線がある時、
長さ$n$の有向道(directed path)という。
また、長さ$n$の有向道で、$\alpha_1 = \alpha_{n+1}$となるものを巡回閉路(cycle)という。
一方で、巡回閉路のない有向グラフは非巡回的(acyclic)であるという。
本論文では、非巡回有向グラフ(DAG)のみを扱う。

頂点$j$から$k$への矢線がある時、$j$を$k$の親(parent)といい、$k$を$j$の子(child)という。
また、$(j,k)\in E$であるすべての頂点$j$からなる集合を$Pa(k)$と表記する。
頂点$j$から$k$への有向道がある時、$j$を$k$の祖先(ancestor)、
$k$を$j$の子孫(descendant)という。
頂点$k$のすべての祖先からなる集合を$An(k)$、すべての子孫からなる集合を$De(k)$と表記する。
また、すべての頂点から$k$と$k$の子孫を除いたものを、$k$の非子孫(non-descendant)といい、
その集合を$\mathit{Nd}(k) \equiv V \backslash (\{k \} \cup De(k))$と表記する。
さらに、因果順序(causal oredering)について定義する。
因果順序とは、その順序に従って変数を並び替えると、
すべての矢線$(j,k)\in E$について、$k$が$j$の原因になることがない順序のことであり、
$\pi =(\pi_1, \dots, \pi_p)$と表記する。
DAGで表現される因果グラフには、このような順序が(一意とは限らないが)存在するという特徴がある。
つまり、因果グラフを同定することは、因果順序を同定することとスケルトンを同定することという
2つの工程に分解することができる。

DAG $G$における頂点上の標本空間$\mathcal X_V$の確率分布に従う
確率変数の集合$X \equiv (X_j)_{j \in V}$について考える。
ここで、確率変数ベクトル$X$は、同時確率密度関数$f_G(X)=f_G(X_1, X_2, \dots, X_p)$で与えられていると仮定する。
$V$の任意の部分集合$S$について、$X_S \equiv \{X_j:j\in S \subset V \}$と
$\mathcal X_S \equiv \times_{j \in S} \mathcal X_j$を定義する。
ただし、$\mathcal X_j$は$X_j$の確率空間である。
また、任意の頂点$j\in V$について、
確率変数ベクトル$X_S$を与えたときの変数$X_j$の
条件付き確率を$f_j(X_j|X_S)$と表記する。
すると、DAG $G$によるモデルは因果マルコフ条件により以下のように因数分解することができる
\cite{Pearl2009-oh}。

\begin{equation}
  f_G(X)=f_G(X_1, X_2, \dots, X_p) = \prod_{j=1}^p f_j(X_j | X_{Pa(j)})
  \label{eq:factorization}
\end{equation}

ここで、$f_j(X_j | X_{Pa(j)})$は、$X_j$の
親変数$X_{Pa(j)} \equiv \{ X_k:k\in Pa(j) \subset V \}$を与えた条件付き確率である。

また、本論文で扱う因果モデルには、因果極小性(causal minimality)を仮定する。
因果極小性とは、DAG $G$で表現される因果構造に従って生成された分布$f_G(X)$は、
$G$の部分グラフにおいては因果マルコフ条件を満たさないことを言う\cite{Zhang2011-da}。
つまり、全ての頂点$j \in V$とその親の1つ$k \in Pa(j)$について、以下が成り立つ。

\begin{equation}
  \forall Pa(j)\backslash\{k\} \subset S \subset Nd(k) \backslash\{k\},
  \quad X_j \notindep X_k | X_S
  \label{minimality}
\end{equation}

最後に、本論文では観察データから因果グラフを同定するという問題を扱うため、
因果グラフの識別可能性について定義する。
識別可能性を直感的に説明すると、条件付き確率分布$f_j(X_j|X_{Pa(j)})$に対してある仮定を置くと、
同時確率密度関数$f_G(X)$を与えたDAG $G$の構造を一意に決定付けることができるということである。

識別可能性について詳細に定義するために、すべての$j \in V$に関する
条件付き確率分布$f_j(X_j|X_{Pa(j)})$の集合を$\mathcal P$と表記する。
また、DAG $G=(V,E)$について、DAG $G$に関する同時分布のクラスと、
分布$\mathcal P$のクラスを以下で定義する。

\begin{equation}
  \mathcal F(G;\mathcal P) \equiv \{ f_G(X) = \prod_{j \in V} f_j(X_j|X_{Pa(j)}) ;
  \text{ここで、 } f_j(X_j|X_{Pa(j)}) \in \mathcal P \quad \forall j \in V \}
\end{equation}

続いて、$p$個の変数からなるDAGの集合を$\mathcal G_p$と表記する。
そこで、DAG $\mathcal G_p$の空間上の確率分布のクラス$\mathcal P$における
識別可能性を以下のように定義する。

\begin{df}[識別可能性]
  条件付き分布のクラス$\mathcal P$が$\mathcal G_p$において識別可能であるとは、
  $G,G' \in \mathcal G_p$において$G \neq G'$であるならば、
  $f_G = f_{G'}$を満たすような$f_G \in \mathcal F(G; \mathcal P)$と
  $f_G' \in \mathcal F(G'; \mathcal P)$が存在しないことである。
\end{df}
