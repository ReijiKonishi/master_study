%!TEX root = ../thesis.tex

\section{既存モデルとその識別可能性}
\label{part:model}

本章ではまず、本論文で用いる数学記号を導入し、
非巡回有向グラフ(Directed Acyclic Graph, DAG)モデルとその識別可能性を定義する。
その後、本論文の提案モデルを構成する2つの既存モデルについて概説する。
既存モデルの1つ目は、連続変数を扱うDAGモデルで、
Shimizu \textit{et al.}(2006)\cite{Shimizu2006-yu}によって提案された
線形非ガウス非巡回モデル(Linear Non-Gaussian Acyclic Model; LiNGAM)である。
2つ目は、主に離散変数を扱うDAGモデルで、
Park and Raskutti(2017)\cite{Park2017-hw}によって提案された
2次分散関数(Quadratic Variance Function, QVF)DAGモデルである。
QVF-DAGモデルの識別可能性は、Park and Raskutti(2017)\cite{Park2017-hw}によって
過分散スコアを用いて証明されたが、
本論文ではPark and Park(2019)\cite{Park2019-qy}が提案した
モーメント比スコアを拡張することで、識別可能条件の緩和を行う。


%
%!TEX root = ../thesis.tex

本章ではまず、本論文で用いる数学記号を導入し、
非巡回有向グラフ(Directed Acyclic Graph, DAG)モデルを定義する。
その後、既存研究として、2次分散関数(Quadratic Variance Function)DAGモデル\cite{Park2017-hw}と、
混合因果モデル(Mixed Causal Model)\cite{Wenjuan2018-nm}について概説し、
本論文で提案するモデルの詳細について述べる。

\subsection{数学的準備}

グラフは頂点(node)の集合$V=\{1,2,\dots,p\}$と、
頂点同士をつなぐ辺(edge)の集合$E\subset V \times V$ によって、
$G=(V,E)$と表現される。
グラフの辺は有向辺(矢線)と無向辺(双方向矢線)に分けることができ、
2つの頂点$j,k\in V$において、
$(j,k)\in E$かつ$(k,j)\notin E$のとき、$j$から$k$への矢線があるという。
これを$j \rightarrow k$と表現することもある。
一方で、$(j,k)\in E$かつ$(k,j)\in E$のとき、$j$と$k$の間に双方向矢線があるという。
すべての辺が有向辺であるグラフを有向グラフ(directed graph)という。
本論文では、特に断りのない限り、頂点$j$から$k$への矢線がある場合、
$j$が$k$の原因であるといった因果関係があることを表すとする。
つまり、本論文で扱うグラフにおける矢線の有無は因果関係の有無を表しており、
矢線の始点が原因で、矢線の終点が結果である。
このような定性的な因果関係を表すグラフを因果グラフ(causal graph)という。
また、グラフ$G$からすべての矢印を取り除くことによって得られるグラフを
$G$のスケルトンという。

頂点の系列$\alpha_1, \alpha_2, \dots, \alpha_{n+1}$について、
すべての$i=1,2,\dots, n$で、
$\alpha_i \rightarrow \alpha_{i+1}$、または$\alpha_{i+1} \rightarrow \alpha_i$となる矢線がある時、
長さ$n$の道(path)という。
特に、すべての$i=1,2,\dots, n$で、$\alpha_i \rightarrow \alpha_{i+1}$となる矢線がある時、
長さ$n$の有向道(directed path)という。
また、長さ$n$の有向道で、$\alpha_1 = \alpha_{n+1}$となるものを巡回閉路(cycle)という。
一方で、巡回閉路のない有向グラフは非巡回的(acyclic)であるという。
本論文では、非巡回有向グラフ(Directed Acyclic Graph; DAG)のみを扱う。

頂点$j$から$k$への矢線がある時、$j$を$k$の親(parent)といい、$k$を$j$の子(child)という。
また、$(j,k)\in E$であるすべての頂点$j$からなる集合を$Pa(k)$と表記する。
頂点$j$から$k$への有向道がある時、$j$を$k$の祖先(ancestor)、
$k$を$j$の子孫(descendant)という。
頂点$k$のすべての祖先からなる集合を$An(k)$、すべての子孫からなる集合を$De(k)$と表記する。
また、すべての頂点から$k$と$k$の子孫を除いたものを、$k$の非子孫(non-descendant)といい、
その集合を$Nd(k) \equiv V \backslash (\{k \} \cup De(k))$と表記する。
さらに、因果的順序(causal oredering)について定義する。
因果的順序とは、その順序に従って変数を並び替えると、
すべての矢線$(j,k)\in E)$について、$k$が$j$の原因になることがない順序のことであり、
$\pi =(\pi_1, \dots, \pi_p)$と表記する。
DAGで表現される因果グラフには、このような順序が(一意とは限らないが)存在するという特徴がある。
つまり、因果グラフを同定することは、因果的順序を同定することとスケルトンを同定することという
2つの工程に分解することができる。

有向グラフ$G$における頂点上の標本空間$\mathcal X_V$の確率分布に従う
確率変数の集合$X \equiv (X_j)_{j \in V}$について考える。
ここで、確率変数ベクトル$X$は、同時確率密度関数$P(G)=P(X_1, X_2, \dots, X_p)$で与えられていると仮定する。
$V$の任意の部分集合$S$について、$X_S \equiv \{X_j:j\in S \subset V \}$と
$\mathcal X_S \equiv \times_{j \in S} \mathcal X_j$を定義する。
ただし、$\mathcal X_j$は$X_j$の確率空間である。
また、任意の頂点$j\in V$について、
確率変数ベクトル$X_S$を与えたときの変数$X_j$の
条件付き確率を$P(X_j|X_S)$と表記する。
すると、DAG $G$によるモデルは以下のように因数分解することができる\cite{Pearl2009-oh}。

\begin{equation}
  P(G)=P(X_1, X_2, \dots, X_p) = \prod_{j=1}^p (X_j | X_{Pa(j)})
\end{equation}

ここで、$(X_j | X_{Pa(j)})$は、$X_j$の
親変数$X_{Pa(j)} \equiv \{ X_k:k\in Pa(j) \subset V \}$を与えた条件付き確率である。



\subsection{2次分散関数 (QVF) DAGモデル}

本節では、Park and Raskutti(2017)\cite{Park2017-hw}によって提案された
2次分散関数 (QVF) DAGモデル

%
%!TEX root = ../thesis.tex

\subsection{LiNGAM}

本節では、連続変数データから因果構造を推定するモデルとして、
LiNGAM\cite{Shimizu2006-yu}について概説する。

LiNGAMは、観測データがDAGによって表現されるデータ生成過程から生成されるものと仮定する。
$p$個の観測変数$X = \{X_1, \dots, X_p \}$に対するLiNGAMは以下のように書ける。

\begin{equation}
  X_j = e_j + b_{j0} + \sum_{k \in Pa(j)} b_{jk}X_j
  \quad \text{with} \quad e_j \sim \text{non-Gaussian}
  \label{eq:lingam}
\end{equation}

各観測変数$X_j$は、その変数以外の観測変数$X_k$とその誤差変数$e_j$の線形和である。
また、それぞれの係数$b_{jk}$は、変数$X_k$から変数$X_j$への直接的な因果効果の大きさを表す。
加えて、誤差変数$e_j$はすべて平均0、分散非ゼロの非ガウス分布に従う連続確率変数であり、
お互いに独立であることを仮定する。
つまり、潜在交絡変数が存在しないことを仮定する\cite{Spirtes2000-mf}。

LiNGAMはShimizu \textit{et al.}(2006)\cite{Shimizu2006-yu}によって、
因果グラフとグラフを構成するパラメータが一意に識別可能であることが証明されている。

%
%!TEX root = ../thesis.tex

\subsection{2次分散関数 (QVF) DAGモデル}

本節では、Park and Raskutti(2017)\cite{Park2017-hw}によって提案された
2次分散関数 (QVF) DAGモデルについて概説する。

QVF-DAGモデルは、各頂点の親による条件付き分布$\mathcal P$の分散が、
平均の2次式で与えられているというモデルであり、
以下のように定義される。

\begin{df}[QVF-DAGモデル\cite{Park2017-hw}]
2次分散関数(Quadratic variance function, QVF)DAGモデルは、
各頂点の親による条件付き分布の分散が、平均の2次式で表現できるDAGモデルである。
つまり、すべての$j \in V$について、以下を満たすような$\beta_{j0}, \beta_{j1} \in \mathbb R$が存在する。
\begin{equation}
\mathit{Var}(X_j|X_{Pa(j)}) = \beta_{j0} E(X_j | X_{Pa(j)}) + \beta_{j1} E(X_j | X_{Pa(j)})^2
\end{equation}
\end{df}

%
%!TEX root = ../thesis.tex

\subsection{QVF-DAGモデルの識別可能性}

本節ではQVF-DAGモデルが識別可能であることを証明する。
QVF-DAGモデルの識別可能性は
Park and Raskutti(2017)\cite{Park2017-hw}によって初めて証明されたが、
本論文ではPark and Park(2019)\cite{Park2019-qy}のアイデアを用いることにより、
識別可能条件の緩和も行う。

まず初めに、QVF-DAGモデルにおけるモーメント(積率)について
以下のような関係性が成立していることを示し、識別可能性の証明に利用する。

\begin{prop}
  リンク関数$(g_j(X_{Pa(j)}))_{j \in V}$が非退化であるQVF-DAGモデル~\eqref{QVF}において、
  任意の頂点$j \in V$、任意の集合$S_j \subset \mathit{Nd}(j)$に関して、
  以下のモーメント関係が成立している。
  \begin{equation}
    \frac{E(X_j^2)}
    {E \left[ \beta_0 E(X_j | X_{S_j}) + (\beta_1 + 1)E(X_j | X_{S_j})^2 \right]}
    \geq 1
  \end{equation}
  同様に、
  \begin{equation}
    E(\mathit{Var}( E(X_j | X_{Pa(j)}) | X_{S_j} )) \geq 1
  \end{equation}
  等号成立は、$S_j$が頂点$j$の親変数すべてを含むとき($Pa(j)\subset S_j$)である。
\end{prop}

\begin{proof}
  分散とモーメントの関係性と、2次分散関数性の定義を利用すると、
  2次分散関数性を満たす確率変数$X$のモーメントについて、以下の関係性が成り立つ。
  \begin{alignat*}
    \mathit{Var}(X) &= E(X^2) - E(X)^2 & \qquad & \text{分散の公式より} \\
                    &= \beta_0 E(X) + \beta_1 E(X)^2 && \text{2次分散関数性の定義より}
  \end{alignat*}
  よって、
  \begin{equation*}
    E(X^2) &= \beta_0 E(X) + (\beta_1 + 1) E(X)^2
  \end{equation*}

  ここで、記号の簡単のために、$f(\mu) = \beta_0 \mu + (\beta_1 + 1)\mu^2$と関数を定義する。
  すると、任意の頂点$j \in V$、任意の空でない集合$S_j \subset \mathit{Nd}(j)$について、
  以下のように書ける。
  \begin{equation}
    \begin{split}
      E(X_j^2 | S_j) &= E(E(X_j^2 | X_{Pa(j)}) | S_j) \\
                     &= E(f(E(X_j | X_{Pa(j)})) | S_j)
      \label{moment_related}
    \end{split}
  \end{equation}

  イェンセンの不等式と関数$f(\cdot)$が凸であることを利用すると、以下が導ける。
  \begin{equation}
    \begin{split}
      E(f(E(X_j | X_{Pa(j)})) | S_j) & \geq
      f(E(E(X_j | X_{Pa(j)}) | S_j)) \\
      &= f(E(X_j | S_j))
      \label{Jensen}
    \end{split}
  \end{equation}

  ここで、モデルの定義より、$E(X_j | X_{Pa(j)}) = g_j(X_{Pa(j)})$であり、
  関数$g_j(\cdot)$は非退化であることを利用すると、
  等号は$S_j$が頂点$j$の親変数すべてを含むとき
  ($Pa(j) \subset S_j \subset \mathit(Nd)(j)$)のみ成立する。

  式~\eqref{moment_related}と式~\eqref{Jensen}を整理すると、
  \begin{equation*}
    \begin{split}
      E(X_j^2 | S_j) - f(E(X_j | S_j)) & \geq 0 \\
      E(X_j^2 | S_j) - \bigl( \beta_0 E(X_j | S_j) +
      (\beta_1 + 1) E(X_j | S_j)^2 \bigl) & \geq 0
    \end{split}
  \end{equation*}
  となり、
  さらに期待値を取ることで、
  \begin{equation*}
    E(X_j^2) - E\bigl( \beta_0 E(X_j | S_j) +
    (\beta_1 + 1) E(X_j | S_j)^2 \bigl) & \geq 0
  \end{equation*}
  が得られる。

  よって、以下が成り立つ。
  \begin{equation*}
    \frac{E(X_j^2)}
    {E\bigl( \beta_0 E(X_j | S_j) + (\beta_1 + 1) E(X_j | S_j)^2 \bigl)}
    \geq 1
  \end{equation*}
\end{proof}

