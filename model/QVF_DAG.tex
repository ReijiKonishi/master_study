%!TEX root = ../thesis.tex

\subsection{2次分散関数 (QVF) DAGモデル}

本節では、Park and Raskutti(2017)\cite{Park2017-hw}によって提案された
2次分散関数 (QVF) DAGモデルについて概説する。

QVF-DAGモデルは、各頂点の親による条件付き分布$\mathcal P$の分散が、
平均の2次式で与えられているというモデルであり、
以下のように定義される。

\begin{df}[QVF-DAGモデル\cite{Park2017-hw}]
2次分散関数(Quadratic variance function, QVF)DAGモデルは、
各頂点の親による条件付き分布の分散が、平均の2次式で表現できるDAGモデルである。
つまり、すべての$j \in V$について、以下を満たすような$\beta_{j0}, \beta_{j1} \in \mathbb R$が存在する。
\begin{equation}
\mathit{Var}(X_j|X_{Pa(j)}) = \beta_{j0} E(X_j | X_{Pa(j)}) + \beta_{j1} E(X_j | X_{Pa(j)})^2
\end{equation}
\end{df}
