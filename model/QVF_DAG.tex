%!TEX root = ../thesis.tex

\subsection{2次分散関数 (QVF) DAGモデル}

本節では、Park and Raskutti(2017)\cite{Park2017-hw}によって提案された
2次分散関数 (QVF) DAGモデルについて概説する。

QVF-DAGモデルは、各頂点の親による条件付き分布$\mathcal P$の分散が、
平均の2次式で与えられているというモデルであり、
以下のように定義される。

\begin{df}[QVF-DAGモデル\cite{Park2017-hw}]
2次分散関数(Quadratic variance function, QVF)DAGモデルは、
各頂点の親による条件付き分布の分散が、平均の2次式で表現できるDAGモデルである。
つまり、すべての$j \in V$について、以下を満たすような$\beta_{j0}, \beta_{j1} \in \mathbb R$が存在する。
\begin{equation}
\mathit{Var}(X_j|X_{Pa(j)}) = \beta_{j0} E(X_j | X_{Pa(j)}) + \beta_{j1} E(X_j | X_{Pa(j)})^2
\end{equation}
\end{df}

2次分散関数による確率分布は、Morris(1982)\cite{Morris1982-nc}によって
正準型指数分布族の文脈において導入され、
ポアソン分布、二項分布、負の二項分布、ガンマ分布が含まれる。
DAGモデルの文脈においては、各頂点の分布がその頂点の親集合からの影響を受けていると考えるため、
各頂点の親による条件付き分布は以下のように記述することができる。

\begin{equation}
  P(X_j|X_{Pa(j)}) = \exp \left( \theta_{jj}X_j  + \sum_{(k,j)\in E} \theta_{jk}X_k X_j -
  B_j(X_j) - A_j \left( \theta_{jj} + \sum_{(k,j) \in E} \theta_{jk} X_k \right) \right)
\end{equation}

ここで、$A_j(\cdot)$は対数分配関数(log-partition function)、
$B_j(\cdot)$は指数分布族によって決まる関数、
$\theta_{jk}\in \mathbb R$は頂点$j$に対応するパラメータである。
DAGモデルの因数分解(\ref{eq:factorization})式により、QVF-DAGモデルの同時確率分布は、
以下のように記述することができる。

\begin{equation}
  P(X) = \exp \left( \sum_{j\in V} \theta_{jj}X_j + \sum_{(k,j)\in E} \theta_{jk}X_k X_j
  - \sum_{j \in V} B_j(X_j) - \sum_{j \in V} A_j
  \left( \theta_{jj} + \sum_{(k,j)\in E} \theta_{jk} X_k \right)\right)
\end{equation}
