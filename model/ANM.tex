%!TEX root = ../thesis.tex

\subsection{Additive Noise Model}

本節では、連続変数データを扱うDAGモデルとして、
広く用いられているAdditive Noise Model(ANM)
\cite{Shimizu2006-yu}
\cite{Hoyer2008-oo}
\cite{Peters2013-eb}
\cite{Peters2014-ro}
\cite{Park2020-ey}
とその識別可能性について概説する。

ANMは、観測変数の同時分布が、
以下の構造方程式と誤差から生成されるDAGモデルである。

\begin{equation}
  X_j = f_j(X_{Pa(j)}) + e_j , \quad e_j \sim (0, \sigma_j^2)
  \label{def:ANM}
\end{equation}

ここで、$(f_j)_{j\in V}$は任意の関数であり、
$(e_j)_{j\in V}$は平均ゼロでそれぞれ異なる分散$(\sigma_j^2)_{j\in V}$に従う
互いに独立な確率変数である。

また、ANMの特殊形として、$(f_j)_{j\in V}$が全て線形な関数で記述されるモデルを、
線形構造方程式モデル(SEM)という。
つまり、観測変数の同時分布が以下の線形方程式で定義されるDAGモデルである。

\begin{equation}
  X_j = \theta_j + \sum_{k \in Pa(j)} \theta_{jk} X_k + e_j, \quad e_j \sim (0, \sigma_j^2)
\end{equation}

ここで、それぞれの係数$\theta_{jk}$は、DAG $G$における頂点$k$から頂点$j$の
直接的な因果効果の大きさを表す。
つまり、頂点$k$が頂点$k$の親であるときは$\theta_{jk} \neq 0$であり、
それ以外のときは$\theta_{jk}=0$である。

これらのDAGモデルは、関数形や誤差変数の分布についていくつかの制約を課すと、
識別可能であることが証明されている。
代表的な識別可能条件について以下に簡単にまとめる。

\begin{itemize}
  \item
  全ての関数$(f_j)_{j \in V}$が非線形である非線形ANM\cite{Hoyer2008-oo}

  \item
  全ての関数$(f_j)_{j \in V}$が線形であり、
  観測変数$(X_j)_{j \in V}$または誤差変数$(e_j)_{j\in V}$のいずれかの確率分布が、
  非ガウス分布に従う線形非ガウス非巡回モデル(LiNGAM)\cite{Shimizu2006-yu}

  \item
  全ての関数$(f_j)_{j \in V}$が線形であり、
  誤差変数$e_j$の分散が全て等しい、または全て既知である線形ANM\cite{Peters2013-eb}
\end{itemize}
