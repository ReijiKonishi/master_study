\documentclass[11pt,uplatex]{jsarticle}
\usepackage{amsmath,amssymb}
\usepackage{float}
\usepackage[scaled]{helvet}
\usepackage{amsthm}
\usepackage{color}
\usepackage[dvipdfmx]{graphicx}
\usepackage{enumitem}
\usepackage{algorithm}
\usepackage{algorithmic}
\usepackage[dvipdfmx]{graphicx}
\usepackage{array}
\usepackage{booktabs}

%余白の設定
\usepackage[top=25truemm,
bottom=30truemm,
left=25truemm,
right=25truemm]{geometry}
%行間の設定
\renewcommand{\baselinestretch}{1.2}

% 定理環境
\theoremstyle{definition}
\newtheorem{df}{定義}[section]
\newtheorem{theo}[df]{定理}
\newtheorem{prop}[df]{命題}
\newtheorem{lemm}[df]{補題}
\newtheorem{cor}[df]{系}
\newtheorem{ex}[df]{例}
\newtheorem{fact}[df]{事実}
\newtheorem{nex}[df]{反例}

\newcommand{\indep}{\mathop{\,\perp\!\!\!\!\!\!\perp\,}}
\newcommand{\notindep}{\mathop{\,\perp\!\!\!\!\!/\!\!\!\!\!\perp\,}}
\renewcommand{\algorithmicrequire}{\textbf{Input:}}
\renewcommand{\algorithmicensure}{\textbf{Output:}}
\newcommand{\argmax}{\mathop{\rm arg~max}\limits}
\newcommand{\argmin}{\mathop{\rm arg~min}\limits}

%
% \qed を自動で入れない proof 環境を再定義
%
\makeatletter
\renewenvironment{proof}[1][\proofname]{\par
  \normalfont
  \topsep6\p@\@plus6\p@ \trivlist
  \item[\hskip\labelsep{\bfseries #1}\@addpunct{\bfseries.}]\ignorespaces
}{%
  \endtrivlist
}
\renewcommand{\proofname}{証明}
\newcommand{\relmiddle}[1]{\mathrel{}\middle#1\mathrel{}}
\makeatother

\makeatletter
\renewcommand{\ALG@name}{提案アルゴリズム}
\makeatother

%\title{マーケティング・リサーチにおける \\ 統計的因果探索を用いた因果仮説構築に関する研究}
%\author{データサイエンス研究科, 株式会社マクロミル \\ 小西 伶児}
%\date{\today}


\begin{document}
%
%!TEX root = ../thesis.tex

\maketitle
%
\input{frontmatter/abstract}
%
\tableofcontents
%
%\listoffigures
%
%\listoftables
%

%
\section{序論}
\label{part:introduction}
%
%!TEX root = ../thesis.tex

\subsection{はじめに}


%
\section{モデル}
\label{part:model}
%
%!TEX root = ../thesis.tex

本章ではまず、本論文で用いる数学記号を導入し、
非巡回有向グラフ(Directed Acyclic Graph, DAG)モデルを定義する。
その後、既存研究として、2次分散関数(Quadratic Variance Function)DAGモデル\cite{Park2017-hw}と、
混合因果モデル(Mixed Causal Model)\cite{Wenjuan2018-nm}について概説し、
本論文で提案するモデルの詳細について述べる。

\subsection{数学的準備}

グラフは頂点(node)の集合$V=\{1,2,\dots,p\}$と、
頂点同士をつなぐ辺(edge)の集合$E\subset V \times V$ によって、
$G=(V,E)$と表現される。
グラフの辺は有向辺(矢線)と無向辺(双方向矢線)に分けることができ、
2つの頂点$j,k\in V$において、
$(j,k)\in E$かつ$(k,j)\notin E$のとき、$j$から$k$への矢線があるという。
これを$j \rightarrow k$と表現することもある。
一方で、$(j,k)\in E$かつ$(k,j)\in E$のとき、$j$と$k$の間に双方向矢線があるという。
すべての辺が有向辺であるグラフを有向グラフ(directed graph)という。
本論文では、特に断りのない限り、頂点$j$から$k$への矢線がある場合、
$j$が$k$の原因であるといった因果関係があることを表すとする。
つまり、本論文で扱うグラフにおける矢線の有無は因果関係の有無を表しており、
矢線の始点が原因で、矢線の終点が結果である。
このような定性的な因果関係を表すグラフを因果グラフ(causal graph)という。
また、グラフ$G$からすべての矢印を取り除くことによって得られるグラフを
$G$のスケルトンという。

頂点の系列$\alpha_1, \alpha_2, \dots, \alpha_{n+1}$について、
すべての$i=1,2,\dots, n$で、
$\alpha_i \rightarrow \alpha_{i+1}$、または$\alpha_{i+1} \rightarrow \alpha_i$となる矢線がある時、
長さ$n$の道(path)という。
特に、すべての$i=1,2,\dots, n$で、$\alpha_i \rightarrow \alpha_{i+1}$となる矢線がある時、
長さ$n$の有向道(directed path)という。
また、長さ$n$の有向道で、$\alpha_1 = \alpha_{n+1}$となるものを巡回閉路(cycle)という。
一方で、巡回閉路のない有向グラフは非巡回的(acyclic)であるという。
本論文では、非巡回有向グラフ(Directed Acyclic Graph; DAG)のみを扱う。

頂点$j$から$k$への矢線がある時、$j$を$k$の親(parent)といい、$k$を$j$の子(child)という。
また、$(j,k)\in E$であるすべての頂点$j$からなる集合を$Pa(k)$と表記する。
頂点$j$から$k$への有向道がある時、$j$を$k$の祖先(ancestor)、
$k$を$j$の子孫(descendant)という。
頂点$k$のすべての祖先からなる集合を$An(k)$、すべての子孫からなる集合を$De(k)$と表記する。
また、すべての頂点から$k$と$k$の子孫を除いたものを、$k$の非子孫(non-descendant)といい、
その集合を$Nd(k) \equiv V \backslash (\{k \} \cup De(k))$と表記する。
さらに、因果的順序(causal oredering)について定義する。
因果的順序とは、その順序に従って変数を並び替えると、
すべての矢線$(j,k)\in E)$について、$k$が$j$の原因になることがない順序のことであり、
$\pi =(\pi_1, \dots, \pi_p)$と表記する。
DAGで表現される因果グラフには、このような順序が(一意とは限らないが)存在するという特徴がある。
つまり、因果グラフを同定することは、因果的順序を同定することとスケルトンを同定することという
2つの工程に分解することができる。

有向グラフ$G$における頂点上の標本空間$\mathcal X_V$の確率分布に従う
確率変数の集合$X \equiv (X_j)_{j \in V}$について考える。
ここで、確率変数ベクトル$X$は、同時確率密度関数$P(G)=P(X_1, X_2, \dots, X_p)$で与えられていると仮定する。
$V$の任意の部分集合$S$について、$X_S \equiv \{X_j:j\in S \subset V \}$と
$\mathcal X_S \equiv \times_{j \in S} \mathcal X_j$を定義する。
ただし、$\mathcal X_j$は$X_j$の確率空間である。
また、任意の頂点$j\in V$について、
確率変数ベクトル$X_S$を与えたときの変数$X_j$の
条件付き確率を$P(X_j|X_S)$と表記する。
すると、DAG $G$によるモデルは以下のように因数分解することができる\cite{Pearl2009-oh}。

\begin{equation}
  P(G)=P(X_1, X_2, \dots, X_p) = \prod_{j=1}^p (X_j | X_{Pa(j)})
\end{equation}

ここで、$(X_j | X_{Pa(j)})$は、$X_j$の
親変数$X_{Pa(j)} \equiv \{ X_k:k\in Pa(j) \subset V \}$を与えた条件付き確率である。



\subsection{2次分散関数 (QVF) DAGモデル}

本節では、Park and Raskutti(2017)\cite{Park2017-hw}によって提案された
2次分散関数 (QVF) DAGモデル


%
%!TEX root = ../thesis.tex

\section{QVF-DAGモデルの識別可能性}
\label{part:newidentifiability}

%
%!TEX root = ../thesis.tex

% \subsection{QVF-DAGモデルの識別可能性}

本章では、前章で導入したQVF-DAGモデル\cite{Park2017-hw}が識別可能であることを証明する。
QVF-DAGモデルの識別可能性は
Park and Raskutti(2017)\cite{Park2017-hw}によって
過分散スコア(OverDispersion Score; ODS)を用いて初めて証明された。

過分散とは、ポアソン分布などの分散が期待値に依存する確率分布において、
期待される分散より標本分散の方が大きくなることである。
過分散が生じる原因の1つとして、サンプルの個体差などが挙げられる。
このような個体差などの効果を組み込んだ統計モデルに
一般化線形混合モデル(generalized linear mixed model; GLMM)がある\cite{2012-iq}。

統計的因果探索の文脈において最初に過分散を利用した例は、
Park and Raskutti(2015)\cite{Park2015-tj}のPoisson DAGモデルである。
Park and Raskutti(2015)\cite{Park2015-tj}では
Poisson DAGモデルの各頂点の条件付き分散が過分散であるかどうかを検定することによって、
識別可能性を証明している。
Park and Raskutti(2017)\cite{Park2017-hw}では、
Poisson DAGモデルにおける過分散が、
QVF-DAGモデルでも成立することを利用し、QVF-DAGモデルの識別可能性を証明している。
つまり、Park and Raskutti(2017)\cite{Park2017-hw}による
過分散スコアを用いたQVF-DAGモデルの識別可能性の証明は、
Park and Raskutti(2015)\cite{Park2015-tj}によるPoisson DAGモデルの識別可能性の証明の
自然な拡張であると言える。

Poisson DAGモデルの識別可能条件は、
Park and Park(2019)\cite{Park2019-qy}による
モーメント比スコア(Moment Ratio Score; MRS)を用いた証明も行われている。
モーメント比スコアを用いることによって、
識別可能条件が緩和され、DAGの推定精度やsample complexityも改善することが示されている\cite{Park2019-qy}。

そこで本論文では、Park and Park(2019)\cite{Park2019-qy}による
Poisson DAGモデルのモーメント比スコアを拡張することで、
QVF-DAGモデルの識別可能性を証明する。
そうすることで、従来の識別可能条件\cite{Park2017-hw}を緩和することや、
DAGの推定精度が向上することが期待される。

まず初めに、QVF-DAGモデルにおけるモーメント(積率)について
以下のような関係性が成立していることを示し、識別可能性の証明に利用する。

\begin{prop} \label{prop:MRS}
  リンク関数$(g_j(X_{Pa(j)}))_{j \in V}$が非退化であるQVF-DAGモデル~\eqref{QVF}において、
  任意の頂点$j \in V$、任意の集合$S_j \subset \mathit{Nd}(j)$に関して、
  以下のモーメント関係が成立する。
  \begin{equation}
    \frac{E(X_j^2)}
    {E \left[ \beta_0 E(X_j | X_{S_j}) + (\beta_1 + 1)E(X_j | X_{S_j})^2 \right]}
    \geq 1
    \label{eq:MRS}
  \end{equation}
  同様に、
  \begin{equation}
    E(\mathrm{Var}( E(X_j | X_{Pa(j)}) | X_{S_j} )) \geq 0
  \end{equation}
  等号成立は、$S_j$が頂点$j$の親変数すべてを含むとき($Pa(j)\subset S_j$)である。
\end{prop}

\begin{proof}
  分散とモーメントの関係性と、2次分散関数性の定義を利用すると、
  2次分散関数性を満たす確率変数$X$のモーメントについて、以下の関係性が成り立つ。
  \begin{alignat*}{2}
    \mathrm{Var}(X) &= E(X^2) - E(X)^2 & \qquad & \text{分散の公式より} \\
                    &= \beta_0 E(X) + \beta_1 E(X)^2 && \text{2次分散関数性の定義より}
  \end{alignat*}
  よって、
  \begin{equation*}
    E(X^2) = \beta_0 E(X) + (\beta_1 + 1) E(X)^2
  \end{equation*}

  ここで、記号の簡単のために、関数$f(\mu) = \beta_0 \mu + (\beta_1 + 1)\mu^2$を定義する。
  すると、任意の頂点$j \in V$、任意の空でない集合$S_j \subset \mathit{Nd}(j)$について、
  以下のように書ける。
  \begin{equation}
    \begin{split}
      E(X_j^2 | S_j) &= E(E(X_j^2 | X_{Pa(j)}) | S_j) \\
                     &= E(f(E(X_j | X_{Pa(j)})) | S_j)
      \label{moment_related}
    \end{split}
  \end{equation}

  イェンセンの不等式と関数$f(\cdot)$が凸であることを利用すると、以下が導ける。
  \begin{equation}
    \begin{split}
      E(f(E(X_j | X_{Pa(j)})) | S_j) & \geq
      f(E(E(X_j | X_{Pa(j)}) | S_j)) \\
      &= f(E(X_j | S_j))
      \label{Jensen}
    \end{split}
  \end{equation}

  ここで、モデルの定義より、$E(X_j | X_{Pa(j)}) = g_j(X_{Pa(j)})$であり、
  関数$g_j(\cdot)$は非退化であることを利用すると、
  等号は$S_j$が頂点$j$の親変数すべてを含むとき
  ($Pa(j) \subset S_j \subset \mathit{Nd}(j)$)のみ成立する。

  式~\eqref{moment_related}と式~\eqref{Jensen}を整理すると、
  \begin{equation*}
    \begin{split}
      E(X_j^2 | S_j) &- f(E(X_j | S_j)) \geq 0 \\
      E(X_j^2 | S_j) &- \bigl( \beta_0 E(X_j | S_j) +
      (\beta_1 + 1) E(X_j | S_j)^2 \bigl) \geq 0
    \end{split}
  \end{equation*}
  となり、
  さらに期待値を取ることで、
  \begin{equation}
    E(X_j^2) - E\bigl(\beta_0 E(X_j | S_j) + (\beta_1 + 1) E(X_j | S_j)^2 \bigl) \geq 0
    \label{proposition_1}
  \end{equation}
  が得られる。 よって、以下が成り立つ。
  \begin{equation}
    \frac{E(X_j^2)}
    {E\bigl( \beta_0 E(X_j | S_j) + (\beta_1 + 1) E(X_j | S_j)^2 \bigl)}
    \geq 1
    \label{proposition_2}
  \end{equation}

  ここからは、$E(X_j^2) \geq E\bigl( \beta_0 E(X_j | S_j) +
  (\beta_1 + 1) E(X_j | S_j)^2 \bigl)$ が、
  $E(\mathrm{Var}( E(X_j | X_{Pa(j)}) | X_{S_j} )) \geq 0$と
  同値であることを証明する。
  まず、分散の公式より以下のように書ける。
  \begin{equation}
    \label{variance}
    E(\mathrm{Var}(X_j | S_j))
     = E(E(\mathrm{Var}(X_j | X_{Pa(j)}) | S_j)) + E(\mathrm{Var}(E(X_j | X_{Pa(j)}) | S_j))
  \end{equation}
  ここで、モデルの定義~\eqref{QVF}を式~\eqref{variance}の右辺の第1項目に代入すると、以下のように整理できる。
  \begin{align}
    E(E(\mathrm{Var}(X_j | X_{Pa(j)}) | S_j))
      &= E(E(\beta_0 E(X_j | X_{Pa(j)}) + \beta_1 E(X_j | X_{Pa(j)})^2 | S_j)) \notag \\
      &= E(E(\beta_0 E(X_j | X_{Pa(j)}) | S_j)) + E(E(\beta_1 E(X_j | X_{Pa(j)})^2 | S_j)) \notag \\
      &= \beta_0 E(X_j) + \beta_1 E(X_j)^2
      \label{organize}
  \end{align}
  さらに、式~\eqref{organize}を用いて式~\eqref{variance}を整理すると、以下のように書ける。
  \begin{equation}
    \label{variance_2}
    E(\mathrm{Var}(E(X_j | X_{Pa(j)}) | S_j))
     = E(\mathrm{Var}(X_j | S_j)) - \beta_0 E(X_j) - \beta_1 E(X_j)^2
  \end{equation}

  ここから式~\eqref{variance_2}の右辺を整理すると、以下のように書ける。
  ただし、最後の不等号は式~\eqref{proposition_1}より成立する。
  \begin{align*}
    E(&\mathrm{Var}(X_j | S_j)) - \beta_0 E(X_j) - \beta_1 E(X_j)^2 \\
     &= E(E(X_j^2 | S_j) - E(X_j | S_j)^2) - \beta_0 E(X_j) - \beta_1 E(X_j)^2 \\
     &= E(E(X_j^2 | S_j)) - E(E(X_j | S_j)^2) - E(\beta_0 E(X_j | S_j)) - E(\beta_1 E(X_j | S_j)^2) \\
     &= E(X_j) - E(\beta_0 E(X_j | S_j) + (\beta_1 + 1) E(X_j | S_j)^2) \\
     & \geq 0
  \end{align*}

  よって、式~\eqref{eq:MRS}は、
  $E(\mathrm{Var}( E(X_j | X_{Pa(j)}) | X_{S_j} )) \geq 0$
  と同値である。
  \qed

\end{proof}

ここからは、命題\ref{prop:MRS}が
QVF-DAGモデルの識別可能性に利用できることを直感的に理解するために、
各頂点の親変数による条件付き確率分布がポアソン分布である
2変数DAGモデルを例にその識別可能性を証明する。
そこで、図\ref{fig:ex_bivariate}のようなDAGモデルを考える。

\begin{itemize}
  \item $G_1 \colon X_1 \sim \mathrm{Poisson}(\lambda_1),
         \quad X_2 \sim \mathrm{Poisson}(\lambda_2) \quad$ ただし、$X_1$と$X_2$は独立

  \item $G_2 \colon X_1 \sim \mathrm{Poisson}(\lambda_1),
         \quad X_2|X_1 \sim \mathrm{Poisson}(g_2(X_1))$

  \item $G_3 \colon X_2 \sim \mathrm{Poisson}(\lambda_2),
         \quad X_1|X_2 \sim \mathrm{Poisson}(g_1(X_2))$

  ただし、$g_1$と$g_2$は非退化な任意の関数である。
  $(g_1, g_2 \colon \mathbb{N} \cup \{ 0 \} \rightarrow \mathbb{R}^+)$
\end{itemize}

\begin{figure}[h]
  \centering
  \includegraphics{./picture/bivariate.pdf}
  \caption{2変数のPoisson-DAGモデル}
  \label{fig:ex_bivariate}
\end{figure}

命題\ref{prop:MRS}より、$G_1$におけるすべての頂点$j \in \{ 1,2 \}$について、
$E(X_j^2) = E(X_j) + E(X_j)^2$である。
$G_2$においては、以下が成り立つ。
\begin{equation*}
  E(X_1^2) = E(X_1) + E(X_1)^2, \quad \text{かつ} \quad
  E(X_2^2) > E(X_2) + E(X_2)^2
\end{equation*}
同様に、$G_3$においては、以下が成り立つ。
\begin{equation*}
  E(X_1^2) > E(X_1) + E(X_1)^2, \quad \text{かつ} \quad
  E(X_2^2) = E(X_2) + E(X_2)^2
\end{equation*}
つまり、モーメント比$E(X_j^2) / (E(X_j) + E(X_j)^2)$によって、
真のグラフ構造を同定することが可能である。

命題\ref{prop:MRS}のモーメント比を用いる方法は、
一般的な$p$変数のQVF-DAGモデルにも適用することが可能であり、
モーメント比~\eqref{eq:MRS}が1か1以上かを確かめることで識別可能性を証明することができる。

\begin{theo}[QVF-DAGモデルの識別可能性]
  2次分散関数性を満たす係数$(\beta_{j0}, \beta{j1})_{j=1}^p$が存在し、
  QVF-DAGモデル\eqref{eq:factorization}のクラスについて考える。
  任意の頂点$j \in V$について、$\beta_{j1} > -1$であり、
  リンク関数$g_j(\cdot)$が非退化であるならば、
  QVF-DAGモデルは識別可能である。
\end{theo}

\begin{proof}
  一般性を失わずに、真の因果順序が一意であり、$\pi = (\pi_1, \dots, \pi_p)$であると仮定する。
  また、簡単のために、$X_{1:j} = (X_{\pi_1}, X_{\pi_2}, \dots, X_{\pi_j})$、
  $X_{1:0} = \emptyset$と定義する。
  加えて、モーメント関連関数$f(\mu) = \beta_0 \mu + (\beta_1 + 1)\mu^2$を定義する。
  ここから数学的帰納法を用いてQVF-DAGモデルの識別可能性を証明する。

  \begin{quote}
    \underline{\textbf{Step(1)}} \\
    因果順序が最初である$\pi_1$について、
    命題\ref{prop:MRS}を用いると、
    $E(X_{\pi_1}^2) = E(f(E(X_{\pi_1})))$であるのに対し、
    任意の頂点$j \in V \backslash \{ \pi_1 \}$については、
    $E(X_j ^2) > E(f(E(X_j)))$である。
    よって、因果順序が1番目の要素$\pi_1$を特定することができる。
  \end{quote}

  \begin{quote}
    \underline{\textbf{Step(m-1)}} \\
    因果順序が$(m-1)$番目の要素について、
    因果順序が先の$(m-1)$個の要素とその親が正しく推定されていると仮定する。
    つまり、因果順序が($m-1$)番目の要素については、
    $E(X_{\pi_{m-1}}^2) = E(f(E(X_{\pi_{m-1}} | X_{1:(m-2)})))$が成立していると仮定する。
    一方で、任意の頂点$k \in \{\pi_m, \dots, \pi_p \}$については
    以下が成立していると仮定する。
    \begin{equation*}
      E(X_j^2) > E(f(E(X_j | X_{1:(m-2)})))
    \end{equation*}
  \end{quote}

  \begin{quote}
    \underline{\textbf{Step(m)}} \\
    因果順序が$m$番目の要素とその親について考える。
    帰納法の仮定より、$\pi_m$は、
    $E(X_{\pi_m}^2) = E(f(E(X_{\pi_m} | X_{1:(m-1)})))$である。
    一方で、$j \in \{ \pi_{m+1}, \dots, \pi_p \}$については、
    $E(X_j^2) > E(f(E(X_j | X_{1:(m-1)})))$である。
    よって、因果順序が$m$番目の要素$\pi_m$を特定することができる。

    親変数に関しては、$P(G)$の因数分解\eqref{eq:factorization}による
    以下の条件付き独立関係より導くことができる。
    \begin{align*}
      E(X_{\pi_m}^2) &= E(f(E(X_{\pi_m} | X_{1:(m-1)}))) \\
                     &= E(f(E(X_{\pi_m} | X_{Pa(\pi_m)})))
    \end{align*}
    つまり、上記の関係が成立するような最小の集合を
    $X_{1:(m-1)}$の中から$\pi_m$の親として選択することができる。
  \end{quote}
\qed
\end{proof}

Park and Raskutti(2017)\cite{Park2017-hw}によって証明された
QVF-DAGモデルの識別可能条件には、
$Pa(j) \nsubseteq S_j$のとき、すべての$x \in \mathcal X_{S_j}$について、
$\mathrm{Var}(E(X_j | X_{Pa(j)}) | X_{S_j} = x) > 0$という
仮定が含まれていた。
しかし、本論文における識別可能条件にはそのような仮定は含まれていない。
つまり、従来の識別可能条件\cite{Park2017-hw}を緩和している。
この識別可能条件の緩和によってPoisson-SEMの学習が容易になることが、
Park and Park(2019)\cite{Park2019-qy}の3.2節において議論されている。


%
%!TEX root = ../thesis.tex

\section{提案モデル}
\label{part:proposal}

本章では前章で俯瞰した2つのモデル\cite{Shimizu2006-yu}\cite{Park2017-hw}を用いることによって、
連続変数と離散変数が混合したデータにおけるDAGモデルを提案し、
その識別可能性を証明する。
その後、提案モデルを推定する手法について述べる。

%
%!TEX root = ../thesis.tex

\subsection{提案モデル}

提案モデルにおける変数は、離散変数と連続変数に分けられ、
離散変数は0以上の整数を取る確率変数であると仮定する。
そこで提案モデルを以下のように定義する。

\begin{enumerate}
  \item
  $p$個の観測変数$X = \{ X_1, \dots, X_p \}$は
  DAGによって表現されるデータ生成仮定から生成されており、
  各変数の親変数がその変数の直接的な原因である。

  \item
  連続変数に割り当てられた変数$X_j$は、
  その親変数$Pa(j)$と誤差変数$e_j$の線形和である。

  \begin{equation}
    X_j = e_j + \theta_{j} + \sum_{k \in Pa(j)} \theta_{jk}X_j
    \quad \text{with} \quad e_j \sim \mathit{Laplace}(0, b_j)
    \label{eq:lingam_prop}
  \end{equation}

  それぞれの係数$\theta_{jk}$は、変数$X_k$から変数$X_j$への直接的な因果効果の大きさを表す。
  また、誤差変数$e_j$はラプラス分布に従う確率変数であり、お互いに独立である。

  \item
  離散変数に割り当てられた変数$X_j$は、
  その親変数$Pa(j)$による条件付き確率が、2次分散関数性を満たす。
  つまり、以下を満たすような$\beta_{j0},\beta_{j1} \in \mathbb{R}$が存在する。

  \begin{equation}
    \mathit{Var}(X_j|X_{Pa(j)}) = \beta_{j0} E(X_j | X_{Pa(j)}) + \beta_{j1} E(X_j | X_{Pa(j)})^2
    \label{QVF_prop}
  \end{equation}

  また、各変数の条件付き期待値は、
  その変数の親変数$Pa(j)$と
  任意の単調で微分可能なリンク関数$g_j \colon \mathcal X_{Pa(j)} \rightarrow \mathbb R^+$
  によって以下のように記述される。
  \begin{equation}
    E(X_j | X_{Pa(j)})
    = g_j(X_{Pa(j)})
    = g_j \left(\theta_j + \sum_{k \in Pa(j)} \theta_{jk}X_k \right)
  \end{equation}
\end{enumerate}

%
%!TEX root = ../thesis.tex

\subsection{提案モデルの識別可能性}

本節では、前節で定義したDAGモデルの識別可能性を証明する。
提案モデルは、連続変数と離散変数とが混在することを許容するDAGモデルであるため、
その特殊形として、全てが連続変数であるモデルや全てが離散変数であるモデルを考えることも可能である。
全てが連続変数である場合は、Additive Noise Modelとなり、
モデルの識別可能条件が複数証明されている\cite{Shimizu2006-yu}
\cite{Hoyer2008-oo}
\cite{Peters2013-eb}
\cite{Peters2014-ro}
\cite{Park2020-ey}。
また、全てが離散変数である場合は、QVF-DAGモデル\cite{Park2017-hw}となり、
識別可能性が既に証明されている\cite{Park2017-hw}。
そこで以下では、観測変数集合に連続変数と離散変数の両方が含まれる場合に関する識別可能条件について議論する。
まず、証明の方針について直感的な理解を得るために、
図\ref{fig:prop_three_variate}のような3変数モデルを用いてその識別可能性を示す。
ここで$X,Z$は連続変数、$Y$は離散変数であるとする。
図\ref{fig:prop_three_variate}の3つの因果グラフから生成される分布は、
いずれも$X \indep Z | Y$という条件付き独立関係が成立しており、
因果マルコフ条件のみでは識別できない例である。
しかし、以下で示すように、提案モデルの特徴を利用すると識別することが可能である。

\begin{align*}
  G_1 \colon & X = \theta_{X} + e_X, \quad e_X \sim N(0, \sigma_X^2) \\
             & Y|X \sim \mathit{Poisson}(\lambda), \quad \log(\lambda) = \theta_Y + \theta_{YX}X \\
             & Z = \theta_Z + \theta_{ZY}Y + e_Z, \quad e_Z \sim N(0, \sigma_Z^2)
\end{align*}

\begin{align*}
  G_2 \colon & X = \theta_X + \theta_{XY}Y + e_X, \quad e_X \sim N(0, \sigma_X^2) \\
             & Y|Z \sim \mathit{Possion}(\lambda), \quad \log(\lambda) = \theta_Y + \theta_{YZ}Z \\
             & Z = \theta_{Z} + e_Z, \quad e_Z \sim N(0, \sigma_Z^2)
\end{align*}

\begin{align*}
  G_3 \colon & X = \theta_X + \theta_{XY}Y + e_X, \quad e_X \sim N(0, \sigma_X^2) \\
             & Y \sim \mathit{Poisson}(\lambda) \\
             & Z = \theta_{Z} + e_Z, \quad e_Z \sim N(0, \sigma_Z^2)
\end{align*}

\begin{figure}[ht]
  \centering
  \includegraphics{./picture/prop_three_variate.pdf}
  \caption{3変数のDAGモデル}
  \label{fig:prop_three_variate}
\end{figure}

命題\ref{prop:MRS}より、$G_1, G2$においては
\begin{align*}
  E(Y^2) > E(Y) + E(Y)^2
\end{align*}
である一方で、$G_3$においては
\begin{align*}
  E(Y^2) = E(Y) + E(Y)^2
\end{align*}
となる。
よって、離散変数のモーメント比~\eqref{eq:MRS}が1か1以上かを確かめることで
$G_1, G_2$と$G_3$は識別可能である。

次に$G_1$について、もし連続変数$X, Z$の誤差変数の分散が
$\sigma_X^2 < \sigma_Z^2 + \mathit{Var}(E(Z|Y))$を満たすならば、
全分散の公式を用いて以下が成り立つ。
\begin{align*}
  \mathit{Var}(Z) &= E(\mathit{Var}(Z|Y)) + \mathit{Var}(E(Z|Y)) \\
                  &= \sigma_Z^2 + \mathit{Var}(E(Z|Y)) \\
                  &> \sigma_X^2 \\
                  &= \mathit{Var}(X)
\end{align*}

よって、$X$のほうが因果順序が早いことが分かる。
つまり、誤差変数の分散が$\sigma_X^2 < \sigma_Z^2 + \mathit{Var}(E(Z|Y))$を満たすならば、
因果順序を特定することが可能である。

$G_2$についても同様に、
連続変数$X, Z$の誤差変数の分散が、
$\sigma_Z^2 < \sigma_X^2 + \mathit{Var}(E(X|Y))$を満たすならば、
真の因果順序$\pi = (Z, Y, X)$を特定することが可能である。


ここからは、上記の3変数モデルでの証明の方針を拡張し、
提案モデルが一般的な$p$変数の場合においても識別可能であることを証明する。

まず初めに、提案モデルにおける離散変数に関して、
命題\ref{prop:MRS}と同様の関係が成立していることを示す。

\begin{lemm}
  提案モデルにおいて、
  離散変数が割り当てあられた任意の頂点$j \in D$、任意の集合$S_j \subset Nd(j)$に関して、
  以下のモーメント関係が成立している。
  \begin{equation}
    \frac{E(X_j^2)}
    {E \left[ \beta_0 E(X_j | X_{S_j}) + (\beta_1 + 1)E(X_j | X_{S_j})^2 \right]}
    \geq 1
    \label{prop_MRS}
  \end{equation}
  等号成立は、$S_j$が頂点$j \in D$の親変数全てを含むとき($Pa(j) \subset S_j$)である。
  \label{lem_prop_MRS}
\end{lemm}

\begin{proof}
  提案モデルにおいて、離散変数が割り当てられた頂点は、式~\eqref{QVF_prop}を満たすため、
  任意の変数$X_j \in X_D$について以下の関係が成り立つ。
  \begin{equation*}
    E(X_j^2) = \beta_0 E(X_j) + (\beta_1 + 1)E(X_j)^2
  \end{equation*}

  ここで、記号の簡単の簡単のために、関数$f(\mu) = \beta \mu + (\beta_1 + 1) \mu^2$を定義する。
  すると、任意の頂点$j \in D$、任意の空でない集合$S_j \subset Nd(j)$について、以下のように書ける。
  \begin{equation}
    \begin{split}
      E(X_j^2 | S_j) &= E(E(X_j^2 | X_{Pa(j)}) | S_j) \\
                     &= E(f(E(X_j | X_{Pa(j)})) | S_j)
    \end{split}
    \label{prop_moment}
  \end{equation}
  提案モデルにおいては連続変数と離散変数が混在するモデルを考えているため、
  離散変数が割り当てられた頂点$j \in D$の非子孫の集合$Nd(j)$には、
  連続変数と離散変数の両方が含まれている可能性がある。
  つまり、式~\eqref{prop_moment}は$S_j$に連続変数と離散変数のどちらが含まれていても成立する。

  以降の証明は、命題\ref{prop:MRS}の証明と同様である。

  \qed
\end{proof}

補題\ref{lem_prop_MRS}で証明したモーメント関係は$p$変数の提案モデルでも利用することができる。
つまり、式~\eqref{prop_MRS}が1に等しくなる$X_j \in X_D$が存在するかどうかを確認することによって、
因果順序の1番目の変数が離散変数か否かを判断することができ、
離散変数の場合はその変数を特定することができる。

\begin{theo}[提案モデルの識別可能性]
  \label{theo:prop_identifiability}
  定義\ref{prop_model}によって定義されるDAGモデルは、以下の仮定を満たすとき識別可能である。
  ここで、$\pi$はDAG $G$における因果順序を表す。
  \begin{enumerate}[label=(\Alph*)]
    \item
    連続変数が割り当てられた任意の頂点$j = \pi_m \in C, k \in De(j) \subset C$の
    データ生成過程における誤差変数の分散について、以下が満たされている。
    \begin{equation*}
      \sigma_j^2 < \sigma_k^2 + E(\mathit{Var}(E(X_k | X_{Pa(k)}) | X_{\pi_1}, \dots, X_{\pi_{m-1}}))
    \end{equation*}

    \item
    離散変数が割り当てあられた任意の頂点$j \in D$について、
    $\beta_{j1} > -1$が満たされている。
  \end{enumerate}
\end{theo}

仮定(B)は、ベルヌーイ分布や多項分布によるDAGモデルを除外するための仮定である。
なぜななら、ベルヌーイ分布や多項分布によるDAGモデルは識別不能であることが知られているためである\cite{Heckerman1995-es}。

以下では、定理\ref{theo:prop_identifiability}を証明する。

\begin{proof}
  一般性を失わずに、DAG $G$における因果順序が一意であり、$\pi = (\pi_1, \dots, \pi_p)$であると仮定する。
  また、簡単のために、$X_{1:j} = (X_{\pi_1}, X_{\pi_2}, \dots, X_{\pi_j})$、
  $X_{1:0} = \emptyset$と定義する。
  DAG $G$において、
  連続変数に割り当てられた変数からなる頂点の集合を$C$、
  離散変数に割り当てられた変数からなる頂点の集合を$D$とする。
  加えて、モーメント関連関数$f(\mu) = \beta_0 \mu + (\beta_1 + 1)\mu^2$を定義する。
  ここから数学的帰納法を用いて提案モデルの識別可能性を証明する。

  \begin{quote}
    \textbf{Step(1)}
    \begin{enumerate}[label=(\roman*)]
      \item
      \underline{$\pi_1 = j \in D$の場合} \\
      補題\ref{lem_prop_MRS}より、$E[X_{\pi_1}^2] = E[f(E[X_{\pi_1}])]$が成立する。
      一方で、頂点$j \in D \backslash \{\pi_1\}$では、
      $E[X_j^2] > E[f(E[X_j])]$となる。
      そのため、因果順序が1番目の要素$\pi_1$は、
      $E[X_j^2] = E[f(E[X_j])]$となるような$j \in D$である。
      もし、そのような変数が存在しなければ、$X_{\pi_1}$は連続変数である。

      \item
      \underline{$\pi_1 = j \in C$の場合}\\
      定理\ref{theo:prop_identifiability}の仮定(A)より、
      任意の頂点$k \in C \backslash \{\pi_1\}$について、以下が成立する。
      \begin{align*}
        \mathit{Var}(X_{\pi_1}) &= \sigma_{\pi_1}^2 \\
                                &< \sigma_k^2 + \mathit{Var}(E(X_k | X_{Pa(k)})) \\
                                &= E(\mathit{Var}(X_k | X_{Pa(k)})) + \mathit{Var}(E(X_k | X_{Pa(k)})) \\
                                &= \mathit{Var}(X_k)
      \end{align*}
    \end{enumerate}
  \end{quote}

  \begin{quote}
    \textbf{Step(m-1)} \\
    因果順序が$(m-1)$番目の要素について、因果順序が先の$(m-1)$個の要素とその親が正しく推定されていると仮定する。
  \end{quote}

  \begin{quote}
    \textbf{Step(m)} \\
    因果順序が$m$番目の要素とその親について考える。
    \begin{enumerate}[label=(\roman*)]
      \item
      \underline{$\pi_m = j \in D$の場合} \\
      補題\ref{lem_prop_MRS}より、$E[X_{\pi_m}^2] = E[f(E[X_{\pi_m} | X_{1:(m-1)}])]$が成立する。
      一方で、頂点$j \in \{\{ \pi_{m+1}, \dots, \pi_p\} \cap D\}$では、
      $E[X_j^2] > E[f(E[X_j | X_{1:(m-1)}])]$となる。
      そのため、因果順序が$m$番目の要素$\pi_m$は、
      $E[X_j^2] = E[f(E[X_j | X_{1:(m-1)}])]$となるような$j \in D$である。
      もし、そのような変数が存在しなければ、$X_{\pi_m}$は連続変数である。

      \item
      \underline{$\pi_m = j \in C$の場合} \\
      あああ

    \end{enumerate}

    親変数に関しては$P(G)の因数分解$\eqref{eq:factorization}による
    以下の条件付き独立関係より導くことができる。
    \begin{align*}
      E(X_{\pi_m}^2) &= E(f(E(X_{\pi_m} | X_{1:(m-1)}))) \\
                     &= E(f(E(X_{\pi_m} | X_{Pa(\pi_m)})))
    \end{align*}
    つまり、上記の関係が成立するような最小の集合を
    $X_{1:(m-1)}$の中から$\pi_m$の親として選択することができる。
  \end{quote}

  \qed
\end{proof}


%
%!TEX root = ../thesis.tex

\section{推定アルゴリズム}
\label{part:algorithm}

推定アルゴリズムだよ〜

%
%!TEX root = ../thesis.tex

\section{数値実験}
\label{part:experience}

本章では数値実験を行い、定義\ref{prop_model}による提案モデルが
前章の提案アルゴリズムによって推定可能であることを示す。
また、提案アルゴリズムによる提案モデルの推定精度が、既存のDAG推定アルゴリズムより高いことを示す。
最後に、離散変数の因果順序を推定する際の閾値の設定に関しても考察する。
%
%!TEX root = ../thesis.tex

\subsection{設定}

本論文では、サンプルサイズ$n = \{ 100, 250, 500, 1000, 1500, 2000, 2500 \}$、
頂点数$p = \{ 5, 10 \}$、
連続変数の誤差変数の分布 \{正規分布, 一様分布\} であるデータセットを、
提案モデルに従ってそれぞれ100個ずつ生成した。
その他のデータ生成や推定における設定の詳細は以下とした。

\begin{enumerate}
  \setlength{\itemsep}{0.3cm}
  \item
  $p$個の変数をランダムに連続変数と離散変数に割り当てた。
  連続変数に割り当てられる確率は0.6、離散変数に割り当てられる確率は0.4とした。
  また、離散変数はポアソン分布と二項分布のいずれかに確率0.5で割り当てた。
  ただし、二項分布のパラメータは
  Park and Raskutti(2017)\cite{Park2017-hw}と同様に$N_j=4$に固定した。

  \item
  グラフの構造をランダムに生成するために、
  まず全ての要素が0である$p\times p$隣接行列を生成した。
  次に、Kalisch and B\"{u}hlmann(2007)\cite{Kalisch2007-xg}と同様に、
  成功確率$s$に従う独立なベルヌーイ試行に基づいて
  隣接行列の下三角成分を1に置き換えた。
  成功確率$s$は、各変数と隣接する変数の個数の期待値が2または5になるように
  ランダムに設定した
  \footnote{各変数と隣接する変数の個数の期待値は$s(p-1)$である\cite{Kalisch2007-xg}}。
  隣接行列のゼロ・非ゼロパターンは、
  提案モデルにおける変数間の関係性の強さを表すパラメータ$\theta_{jk}$の
  ゼロ・非ゼロパターンと同じである。

  \item
  提案モデルにおける非ゼロのパラメータ$\theta_{jk} \in \mathbb R$は、
  範囲$\theta_{jk} \in [-0.5,-0.2]\cup [0.2,0.5]$からランダムに選択した。
  また、定数項$\theta_j$は、範囲$\theta_j \in [-0.5,0.5]$からランダムに選択した。

  \item
  連続変数のデータ生成過程における誤差変数の分散$\sigma_j^2$は、
  範囲$\sigma_j^2 \in [0.7,0.8]$からランダムに選択した。

  \item
  離散変数の因果順序を推定する際の閾値は1.01とした。

  \item
  各変数の親変数との関係性の強さを推定する際の
  adaptive Lassoの調整パラメータについて、
  $\lambda$は10分割交差検証によって選択した。
  具体的には、「最小の逸脱度(deviance) +標準誤差」を超えない
  最大の$\lambda$を選択した。
  また、$\gamma$は1に固定した。
\end{enumerate}

既存のDAG推定アルゴリズムには、
Greedy Equivalent Search (GES)\cite{Chickering2002-iq}、
Max-Min Hill-Climbing (MMHC)\cite{Tsamardinos2006-qe}、
DirectLiNGAM\cite{Shimizu2011-pd}、
を用いた。
DAGを推定するアルゴリズムは、制約に基づくアプローチ(constraint-based approach)と
スコアに基づくアプローチ(score-based approach)に大別される。
GESはスコアに基づくアプローチであり、
ベイズ情報量規準(Bayesian Information Criterion; BIC)を用いて
モデルの当てはまりが最も良いマルコフ同値類を推定する方法である。
MMHCは制約に基づくアプローチとスコアに基づくアプローチの両方を組み合わせた
ハイブリッドなアルゴリズムである。
具体的には、まず制約に基づくアプローチによって各変数の親候補と子候補を求め、
次にスコアに基づくアプローチによってモデルの当てはまりが最も良いマルコフ同値類を推定する。
1段階目として制約に基づくアプローチによって親候補と子候補を求めることによって、
2段階目のスコアに基づくアプローチの際の探索空間を小さくしていることが特徴である。
DirectLiNGAMはLiNGAM\cite{Shimizu2006-yu}に特化したアルゴリズムであり、
回帰分析と独立性の評価を繰り返すことによって因果順序を推定する。

DAGの推定精度は、適合率(precision)と再現率(recall)によって測定した。
適合率と再現率の算出方法は以下の通りである。
\begin{itemize}
  \item 適合率(precision)
  \begin{equation*}
    \frac{\text{正しく推定された辺の個数}}{\text{推定されたDAGにおける辺の個数}}
  \end{equation*}

  \item 再現率(recall)
  \begin{equation*}
    \frac{\text{正しく推定された辺の個数}}{\text{真のDAGにおける辺の個数}}
  \end{equation*}
\end{itemize}
ただし、GESとMMHCはマルコフ同値類を推定するアルゴリズムであるため、
推定されたグラフの一部の辺は無向辺のままである。
そのため、本論文ではPark and Park(2019)\cite{Park2019-qy}と同様に、
推定されたマルコフ同値類すべてにおける適合率と再現率の平均値で評価する。

%
%!TEX root = ../thesis.tex

\subsection{DAGの推定精度}

各アルゴリズムのDAGの推定精度を図\ref{fig:plot_gaussian}
と\ref{fig:plot_uniform}に示す。
図\ref{fig:plot_gaussian}は
連続変数の誤差変数が正規分布に従うように生成したデータにおける推定精度を、
図\ref{fig:plot_uniform}は一様分布に従うように生成したデータにおける推定精度を示す。
また、生成されたデータの都合上、
稀に一般化線形モデルやadaptive Lassoの推定が収束しない場合があった。
その際は推定結果を欠測とし、精度指標の算出時に除外した。

実験結果より、提案アルゴリズムは既存アルゴリズムと比較して適合率も再現率も高いことが分かる。
また、提案アルゴリズムは他のアルゴリズムと比較して、サンプルサイズが大きくなるほど
有向辺の向きを正しく推定しており、
特に再現率においては既存アルゴリズムとの差が大きくなる傾向が見られる。
このことから、提案モデルは識別可能であり、本論文による提案アルゴリズムによって推定可能であると言える。
一方、DirectLiNGAMの推定精度は、正規分布の場合は適合率も再現率も0.5を下回るのに対し、
一様分布の場合はGESやMMHCの精度と同程度またはそれ以上である。
これは、LiNGAM\cite{Shimizu2011-pd}が誤差変数に対して非正規性を
仮定していることが影響しているためであると考えられる。

提案アルゴリズムにおける適合率と再現率を比較すると、
サンプルサイズが1000を超えると再現率のほうが高いことが読み取れる。
つまり、サンプルサイズが大きい時、
実際には直接的な因果関係があるにもかかわらず因果関係が無いと推定してしまうという偽陰性が少ない
ことを示している。
因果関係に関する仮説を得ようとする際に偽陰性が多いと、
重要な因果関係を見過ごしてしまう可能性が高まる。
そのため、因果関係の仮説を構築する上で、提案アルゴリズムの再現率が高いことは望ましいと言える。

\begin{figure}[H]
  \includegraphics[width=13cm, bb=9 9 358 434]{./picture/plot_gaussian.pdf}
  \caption{連続変数の誤差変数が正規分布に従う提案モデルにおける各アルゴリズムの精度比較}
  \label{fig:plot_gaussian}
\end{figure}

\begin{figure}[H]
  \includegraphics[width=13cm, bb=9 9 358 434]{./picture/plot_uniform.pdf}
  \caption{連続変数の誤差変数が一様分布に従う提案モデルにおける各アルゴリズムの精度比較}
  \label{fig:plot_uniform}
\end{figure}

%
%!TEX root = ../thesis.tex

\subsection{閾値の設定について}

提案アルゴリズムは、式~\eqref{alg:MRS}による
モーメント比スコア$\widehat{\mathcal S}(m,j)$が閾値より小さいかどうかによって
離散変数と連続変数の因果順序を推定している。
そのため、閾値の設定が因果順序の推定精度に影響を与える。
そこで本節では、以下の数値実験を行うことで、
モーメント比スコアの閾値の設定によって因果順序の推定精度がどのように変化するのかを示す。

本節の数値実験では、
頂点数を$p = 5$、
連続変数の誤差変数の分布を一様分布に固定した。
また、因果順序が一意に定まるように、データ生成過程のグラフ構造は
図\ref{fig:threshold}とした。
四角形で表した頂点$X_1,X_3,X_5$に離散変数を割り当て、
各変数の条件付き分布はポアソン分布に従うこととした。
また、各頂点間の関係性の強さのパラメータ$\theta_{jk}$や定数項$\theta_j$は、
\ref{subsection:setup}節に記載の内容と同じとした。

\begin{figure}[ht]
  \centering
  \includegraphics{./picture/threshold.pdf}
  \caption{閾値の設定による因果順序の推定精度を測定する数値実験におけるグラフ構造}
  \label{fig:threshold}
\end{figure}

閾値は $ \{ 1.000, 1.002, \dots, 1.050 \}$とし、
それぞれ100回ずつデータ生成と因果順序の推定を繰り返した。
因果順序の推定精度は、正解率(Accuracy)によって測定した。
正解率の算出方法は以下の通りである。
\begin{itemize}
  \item 正解率(Accuracy)
  \begin{equation*}
    \frac{\text{因果順序が正しく推定された回数}}{\text{繰り返し回数}}
  \end{equation*}
\end{itemize}

\begin{figure}
  \includegraphics[width=13cm, bb=9 9 358 434]{./picture/plot_threshold.pdf}
  \caption{閾値の設定と因果順序の推定精度の関係}
  \label{fig:plot_threshold}
\end{figure}

ああああ


%
%!TEX root = ../thesis.tex

\section{結論}
\label{part:conclusion}

結論だよ〜

%
%!TEX root = ../thesis.tex

\bibliographystyle{jabbrv}
\bibliography{main_bibliography}
\addcontentsline{toc}{section}{参考文献}
%
%!TEX root = ../thesis.tex
%
\section*{謝辞}
\addcontentsline{toc}{section}{謝辞}
%
本研究活動および論文の執筆にあたり、終始適切な助言を賜り、
丁寧に指導してくださった清水昌平先生に感謝申し上げます。
研究を初めた当初は、統計的因果推論に関する知識は全くない状態でしたが、
ゼミにおける輪読会にて真摯にコメントいただいたり、
外部講師を招いた研究会を開催いただいたりしたことで、
分野の理解を深めることができました。
また、副指導教員を引き受けてくださった伊達平和先生には、
中間報告会などにおいて研究内容の意義に対する貴重な助言をいただきました。
感謝申し上げます。
さらに、日頃のゼミにおいて理論に関して丁寧な指導をしてくださった
青山学院大学経営学部の保科架風先生に感謝申し上げます。

そして、社会人派遣学生として送り出してくださった株式会社マクロミルの皆様に感謝いたします。
特に、「社会人を数年間経験してからいつか大学院に進学したい」という自身の希望を聞き入れ、
派遣を推薦してくださった西部君隆さん、
派遣期間中に様々なご支援をいただいた小笠原道明さん、
論文執筆中の諸業務を調整いただいた丸雄太さんに心から感謝いたします。

最後になりましたが、2年間の学生生活を共に切磋琢磨した
データサイエンス研究科修士課程第1期生の皆様、
清水研究室の皆様に感謝いたします。


%
\end{document}
